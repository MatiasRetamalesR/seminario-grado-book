% This is the Reed College LaTeX thesis template. Most of the work
% for the document class was done by Sam Noble (SN), as well as this
% template. Later comments etc. by Ben Salzberg (BTS). Additional
% restructuring and APA support by Jess Youngberg (JY).
% Your comments and suggestions are more than welcome; please email
% them to cus@reed.edu
%
% See https://www.reed.edu/cis/help/LaTeX/index.html for help. There are a
% great bunch of help pages there, with notes on
% getting started, bibtex, etc. Go there and read it if you're not
% already familiar with LaTeX.
%
% Any line that starts with a percent symbol is a comment.
% They won't show up in the document, and are useful for notes
% to yourself and explaining commands.
% Commenting also removes a line from the document;
% very handy for troubleshooting problems. -BTS

% As far as I know, this follows the requirements laid out in
% the 2002-2003 Senior Handbook. Ask a librarian to check the
% document before binding. -SN

%%
%% Preamble
%%
% \documentclass{<something>} must begin each LaTeX document
\documentclass[12pt,twoside]{templates/facsothesis}
% Packages are extensions to the basic LaTeX functions. Whatever you
% want to typeset, there is probably a package out there for it.
% Chemistry (chemtex), screenplays, you name it.
% Check out CTAN to see: https://www.ctan.org/
%%
\ifxetex
  \usepackage{polyglossia}
  \setmainlanguage{spanish}
  % Tabla en lugar de cuadro
  \gappto\captionsspanish{\renewcommand{\tablename}{Tabla}
          \renewcommand{\listtablename}{Índice de tablas}}
\else
  \usepackage[spanish,es-tabla]{babel}
\fi
%\usepackage[spanish]{babel}
\usepackage{graphicx,latexsym}
\usepackage{amsmath}
\usepackage{amssymb,amsthm}
\usepackage{longtable,booktabs,setspace}
\usepackage{chemarr} %% Useful for one reaction arrow, useless if you're not a chem major
\usepackage[hyphens]{url}
% Added by CII
%\usepackage{hyperref}
\usepackage[colorlinks = true,
            linkcolor = blue,
            urlcolor  = blue,
            citecolor = blue,
            anchorcolor = blue]{hyperref}
\usepackage{lmodern}
\usepackage{float}
\floatplacement{figure}{H}
% End of CII addition
\usepackage{rotating}
\usepackage{placeins} % para fijar la posición de las tablas con \FloatBarrier


\usepackage[]{natbib}


% Next line commented out by CII
%\usepackage{biblatex}
%\usepackage{natbib}
% Comment out the natbib line above and uncomment the following two lines to use the new
% biblatex-chicago style, for Chicago A. Also make some changes at the end where the
% bibliography is included.
%\usepackage{biblatex-chicago}
%\bibliography{thesis}


% Added by CII (Thanks, Hadley!)
% Use ref for internal links
\renewcommand{\hyperref}[2][???]{\autoref{#1}}
\def\chapterautorefname{Chapter}
\def\sectionautorefname{Section}
\def\subsectionautorefname{Subsection}
% End of CII addition

% Added by CII
\usepackage{caption}
\captionsetup{width=5in}
% End of CII addition

% \usepackage{times} % other fonts are available like times, bookman, charter, palatino

% Syntax highlighting #22

% To pass between YAML and LaTeX the dollar signs are added by CII
\title{El efecto mitigador del sentido de justicia en la escuela sobre la intención de participación cívica: un estudio multinivel sobre estudiantes chilenos}
\author{Matías Retamales R.}
% The month and year that you submit your FINAL draft TO THE LIBRARY (May or December)
\date{28 de agosto de 2020}
\division{}
\advisor{Profesor guía: Dr.~Juan Carlos Castillo}
\institution{Universidad de Chile}
\degree{Seminario de Grado - Carrera de Sociología}
%If you have two advisors for some reason, you can use the following
% Uncommented out by CII
% End of CII addition

%%% Remember to use the correct department!
\department{}
% if you're writing a thesis in an interdisciplinary major,
% uncomment the line below and change the text as appropriate.
% check the Senior Handbook if unsure.
%\thedivisionof{The Established Interdisciplinary Committee for}
% if you want the approval page to say "Approved for the Committee",
% uncomment the next line
%\approvedforthe{Committee}

% Added by CII
%%% Copied from knitr
%% maxwidth is the original width if it's less than linewidth
%% otherwise use linewidth (to make sure the graphics do not exceed the margin)
\makeatletter
\def\maxwidth{ %
  \ifdim\Gin@nat@width>\linewidth
    \linewidth
  \else
    \Gin@nat@width
  \fi
}
\makeatother

%Added by @MyKo101, code provided by @GerbrichFerdinands

\setlength\parindent{0pt}


% Added by CII

\providecommand{\tightlist}{%
  \setlength{\itemsep}{0pt}\setlength{\parskip}{0pt}}

\Acknowledgements{

}

\Dedication{

}

\Preface{

}

\Abstract{

}

% End of CII addition
%%
%% End Preamble
%%
%
\let\chaptername\relax
\begin{document}
\bibliographystyle{apalike}
% Everything below added by CII
  \maketitle

\frontmatter % this stuff will be roman-numbered
\pagestyle{empty} % this removes page numbers from the frontmatter



%  \hypersetup{linkcolor=black}
  \setcounter{tocdepth}{1}
  \setlength{\parskip}{0pt}
  \tableofcontents

\setlength\parskip{1em plus 0.1em minus 0.2em}





\mainmatter % here the regular arabic numbering starts
\pagestyle{fancyplain} % turns page numbering back on

\hypertarget{resumen}{%
\chapter*{Resumen}\label{resumen}}
\addcontentsline{toc}{chapter}{Resumen}

La actual paradoja de la baja participación dentro de los canales institucionales de representación política y el aumento de la participación por canales alternativos al margen del sistema político, plantea desafíos para la legitimidad de la democracia chilena. Uno de ellos es la desigualdad de voz política, la cual tiene como consecuencia la subrepresentación de los intereses y expectativas ciudadanas de la población más joven y de menor estatus social. Ante tal problema, la socialización política de la escuela con la formación ciudadana, emerge como una opción para promover previamente la participación cívica. No obstante, la estrecha vinculación entre el estatus social de las familias y la intención de participación cívica de los estudiantes, traslada la preocupación a la transmisión intergeneracional de la desigualdad política que perjudica a los estudiantes más desfavorecidos. Es por esto que, en el presente estudio se busca aportar con evidencia sobre la contribución del sentido de justicia en la escuela en la intención de participación cívica de los estudiantes y su efecto mitigador de la brecha generada por el estatus social familiar. Para tal propuesta se trabaja con datos secundarios obtenidos de la encuesta desarrollada por la Agencia de Calidad de la Educación (2017), la cual realiza la evaluación más reciente de la formación ciudadana en Chile. La investigación cuenta con una muestra representativa de estudiantes de 8° básico (n = 7072) anidados dentro de escuelas, por lo que se trabaja con modelos multinivel para descomponer y estimar simultáneamente efectos, tanto a un nivel individual como a un nivel escolar.

\textbf{Palabras clave}: participación cívica, sentido de justicia en la escuela, socialización política, desigualdad política, mitigar.

\hypertarget{introducciuxf3n}{%
\chapter{Introducción}\label{introducciuxf3n}}

La participación cívica entendida como la acción manifiesta a influir en las decisiones políticas del gobierno o en los resultados políticos (Ekman \& Amnå, 2012), tiene profundos cambios en Chile desde el retorno a la democracia en 1990, configurando la paradoja actual del declive de la participación formal y el aumento de la participación activista. Por un lado, la participación formal que ocupa los canales institucionales para influir en las decisiones gubernamentales (Miranda, 2018; Miranda, Castillo \& Sandoval-Hernández, 2017), registra un declive con la participación en elecciones nacionales, siendo una de las más bajas de América Latina (Programa de las Naciones Unidas para el Desarrollo {[}PNUD{]}, 2017a). Ante la ruptura de fondo entre política y sociedad, este declive es síntoma del debilitamiento de los canales institucionales de participación, la crisis de legitimidad del sistema político y la despartidización (Garretón, 2016; Ruiz, 2015; Somma, 2017). Sin embargo, por otro lado, la participación activista que ocupa canales alternativos para influir en los resultados del sistema político (Miranda, 2018; Miranda et al., 2017), se caracteriza por su aumento al alero de protestas desde la década del 2000, cultivando una autonomía respecto a las instituciones tradicionales de representación política (Bidegain, 2015; Roberts, 2016; Somma \& Bargsted, 2015; Somma \& Medel, 2017).

La literatura señala que el aumento de la participación activista es una alternativa para presionar al sistema político con demandas ciudadanas (PNUD, 2019), lo que es considerado como un proceso de (re)politización de la sociedad chilena al margen de la institucionalidad (e.g., Donoso \& von Bulow, 2017; PNUD, 2019; Roberts, 2016). No obstante, la legitimidad de la democracia chilena es puesta en cuestión, precisamente por su baja representación de intereses y expectativas ciudadanas en la política institucional (PNUD, 2019). Esta situación se vuelve aún más crítica por la desigualdad de voz política, en tanto tienen menos voz en el sistema político y en instancias de participación cívica (i.e., en participación formal y activista) aquellos de menor estatus social (e.g., Bargsted, Somma, Campos \& Joignant, 2017; Corvalan \& Cox, 2015; Roberts, 2016). La evidencia señala que sucede lo mismo con la participación cívica de los jóvenes (e.g., Corvalan \& Cox, 2015; Rozas \& Somma, 2020), pero particularmente se encuentran doblemente mal representados por la democracia, ya que además la edad es un factor de desigualdad política que afecta negativamente su participación formal (e.g., Bargsted et al., 2017; Contreras \& Navia, 2013; Toro, 2008).

Producto de la preocupación por los bajos niveles de participación cívica acumulada en la población más joven, un importante cuerpo de investigación se pregunta por como se puede promover previamente su participación, mediante la formación ciudadana en las escuelas. Desde un horizonte normativo, se espera que las escuelas sean un espacio donde se prepare a los estudiantes para participar a futuro, tanto en la comunidad política como en la vida política (Hess \& McAvoy, 2015). Esto se debe a que la escuela es una institución mediadora, en la cual los estudiantes se enfrentan a una `mini-política' y practican lo que significa ser ciudadanos de una organización política más grande (Flanagan, 2016). De ese modo, la escuela es concebida como un espacio de socialización política (Cumsille \& Martínez, 2015; Deimel, Hoskins \& Abs, 2020; Neundorf, Niemi \& Smets, 2016), pues en ella se cristalizan identidades políticas, valores y comportamientos que perdurarán en la vida de los estudiantes (Neundorf \& Smets, 2017). A partir de la adquisición de conocimientos sobre como funciona el sistema político, los estudiantes pueden desarrollar valores y normas democráticas que promuevan una participación ciudadana activa (Youniss \& Levine, 2009).

Múltiples investigaciones en ciencias sociales estudian empíricamente si la formación ciudadana en las escuelas promueve la intención de participación formal y activista de los estudiantes (e.g., Blaskó, Dinis da Costa \& Vera-Toscano, 2019; Deimel et al., 2020; Castillo, Miranda, Bonhomme, Cox \& Bascopé, 2015a; Eckstein \& Noack, 2016; Isac, Maslowski, Creemers \& van der Werf, 2014; Quintelier \& Hooghe, 2013; Treviño, Villalobos, Béjares \& Naranjo, 2019). No solo a partir del mandato explícito del currículum formal, sino también mediante un currículum oculto no intencionado que emerge de la cultura y el clima escolar (Apple, 1980; González, 2018). No obstante, la evidencia enfatiza la relevancia de la socialización política de las familias, demostrando que los estudiantes de familias con menos estatus social tienen una menor intención de participación cívica (Miranda, Bonhomme, Cox \& Bascopé, 2014; Castillo et al., 2015a; Kudrnáč, 2017; Manganelli, Lucidi \& Alivernini, 2015; Miranda, 2018; Neundorf, et al., 2016). Este resultado explicado comúnmente por el modelo de recursos (e.g., Castillo et al., 2014; Castillo et al., 2015a; Miranda, 2018), es conceptualizado como la transmisión intergeneracional de la desigualdad política (Brady, Schlozman \& Verba, 2015; Miranda, 2018; Schlozman, Verba \& Brady, 2012).

La relevancia de los recursos parentales y la evidencia que respalda la transmisión de los resultados políticos a sus hijos (Brady et al.~2015; Castillo et al., 2014; Miranda, 2018; Schlozman et al.~2012), vuelve crucial el rol de socialización política de las escuelas para paliar la desigualdad política en los estudiantes. Si bien existe una vasta evidencia sociológica sobre los factores sociodemográficos implicados en la desigualdad de voz política en población adulta y joven (e.g., Bargsted et al., 2017; Corvalan \& Cox, 2015; Roberts, 2016; Rozas \& Somma, 2020), la evidencia proporcionada por la disciplina es menor cuanto se trata de la desigualdad política en los estudiantes, ante la cobertura puesta en la transmisión del logro educacional o el estudio de la reproducción social (Miranda, 2018). Es por eso que, la línea de investigación que se preocupa por la capacidad del sistema escolar de desafiar el habitus familiar (c.f. Peña, 2015), sigue la premisa de que la escuela no es completamente un espacio de producción social (c.f. Bourdieu \& Passeron, 1977), pues también en ella está la posibilidad de promover la participación cívica en los estudiantes más desfavorecidos (Bonhomme, Cox, Tham \& Lira, 2015).

Recientes investigaciones se preocupan por como mitigar la brecha generada por el vínculo entre el estatus social familiar y la intención de participación cívica de los estudiantes, a partir de la formación ciudadana en la escuela. La evidencia empírica prueba como mitigadores de este vínculo, al clima abierto en el aula, la participación en la escuela, la participación en actividades cívicas, la participación en decisiones escolares y el aprendizaje formal de ciudadanía (Campbell, 2008; Castillo et al., 2015a; Deimel, et al., 2020; Eckstein \& Noack, 2016; Gainous \& Martens, 2012; Hoskins, Janmaat \& Melis, 2017; Kudrnáč, 2017; Neundorf et al., 2016). Sin embargo, a menudo los estudiantes más desfavorecidos tienen un acceso desigual a estos mitigadores que brinda la formación ciudadana (Deimel et al., 2020; Hoskins et al., 2017), lo que incentiva la búsqueda de más oportunidades para reducir la brecha generada por el estatus social familiar.

Dicho lo anterior, hasta ahora es escasa la evidencia sobre la contribución de la justicia percibida por los estudiantes en la escuela, la cual se caracteriza por una operacionalización que se limita a la percepción sobre el trato justo de los profesores (c.f. Eckstein \& Noack, 2016). Plegarse a los aportes teóricos y empíricos que desarrolla la sociología y la psicología social con el sentido de justicia en la escuela (e.g., Resh \& Sabbagh, 2016; Sabbagh \& Resh, 2016; Sabbagh, Resh, Mor \& Vanhuysse, 2006), puede subsanar la anterior limitación al estudiar la distribución meritocrática de las notas como justicia distributiva (Resh, 2010; Resh \& Sabbagh, 2013; Sabbagh et al., 2006; Sabbagh \& Resh, 2016) y la justicia de los profesores en las relaciones con sus estudiantes como justicia relacional (Čiuladienė \& Račelytė, 2016; Claus, Chory \& Malachowski, 2012). En este sentido, la presente investigación tiene por objetivo general \emph{analizar el efecto del sentido de justicia en la escuela sobre la intención de participación cívica de los estudiantes y su efecto mitigador de la brecha generada por el estatus social familiar.}

La relevancia de investigar el sentido de justicia en la escuela, se debe a que es una experiencia escolar significativa para los estudiantes (Dalbert, 2004), al mismo tiempo que estudios clásicos sobre el desarrollo político de los jóvenes señalan que la justicia social es una motivación importante para la participación cívica (Flanagan \& Sherrod, 1998). Al respecto, la literatura señala que el sentido de justicia en la escuela es parte del currículum oculto de las escuelas (Resh, 2010, 2018; Resh \& Sabbagh, 2013, 2014, 2017), pudiendo cumplir un rol latente en la formación ciudadana de los estudiantes. La reciente evidencia empírica muestra resultados alentadores acerca de como contribuye la dimensión de justicia distributiva y relacional en la socialización cívica de los estudiantes (Pretsch \& Ehrhardt-Madapathi, 2018; Resh, 2018; Resh \& Sabbagh, 2013, 2014, 2017), tal como sucede con la evidencia más cercana que prueba en particular el efecto sobre la intención de participación formal y activista (Eckstein \& Noack, 2016; Isac et al., 2014; Quintelier \& Hooghe, 2013; Treviño et al., 2019). Además, existe evidencia favorable sobre el acceso igualitario a la justicia distributiva (Resh, 2010; Resh \& Dalbert, 2007) y los estudios más cercanos sugieren que la justicia relacional puede mitigar la brecha por estatus social familiar (e.g.~Eckstein \& Noack, 2016), lo que abre una valiosa oportunidad para paliar la desigualdad política en los estudiantes.

Dicho de un modo general, al analizar el efecto del sentido de justicia en la escuela sobre la intención de participación cívica de los estudiantes, se busca aportar con evidencia que retome la preocupación por la baja participación acumulada de la población más jóven en Chile, considerando que las intenciones de participación de los estudiantes son un predictor de la participación cívica a futuro (Quintelier \& Blais, 2016). Además, probar si el sentido de justicia en la escuela mitiga el vínculo entre el estatus social familiar y la intención de participación cívica, tiene como fondo la preocupación por la transmisión intergeneracional de la desigualdad política. En un contexto como el chileno, aquello toma relevancia por los crecientes niveles de desigualdad desde el regreso a la democracia (Flores, Sanhueza, Atria \& Mayer, 2019) y el alto nivel de segregación socioeconómica del sistema escolar (Mizala \& Torche, 2012; Santos \& Elacqua, 2016; Valenzuela, Bellei, de los Ríos, 2014). Por lo tanto, se tiene como hipótesis general que \emph{el sentido de justicia en la escuela tendrá un efecto positivo sobre la intención de participación cívica de los estudiantes y logrará mitigar la brecha generada por el estatus social familiar.}

En este estudio se trabaja con modelos lineales jerárquicos o modelos multinivel, a modo de realizar un análisis a nivel individual y a nivel escolar, tal como lo hacen otras investigaciones en el campo dedicadas a estudiar la intención de participación cívica (e.g., Castillo et al., 2015a; Deimel et al., 2020; Isac et al., 2014; Quintelier \& Hooghe, 2013; Treviño et al., 2019). Por lo tanto, los objetivos específicos de la investigación son: a) analizar el efecto de la justicia distributiva a nivel individual y escolar sobre la intención de participación formal y activista de los estudiantes; b) analizar el efecto de la justicia relacional a nivel individual y escolar sobre la intención de participación formal y activista de los estudiantes; y c) analizar el efecto mitigador de la justicia distributiva y relacional a nivel individual, sobre el vínculo entre el estatus social familiar y la intención de participación formal y activista de los estudiantes. En tal sentido, la pregunta central que orienta la investigación es la siguiente: \emph{¿Cómo afecta el sentido de justicia en la escuela sobre la intención de participación cívica de los estudiantes chilenos?}

\hypertarget{participaciuxf3n-cuxedvica-en-chile}{%
\chapter{Participación cívica en Chile}\label{participaciuxf3n-cuxedvica-en-chile}}

Para una democracia plena se necesita una ciudadanía que participe de la toma de decisiones colectivamente vinculantes (PNUD, 2019), a modo de ejercer un control sobre las autoridades e instituciones (Núñez, Osorio \& Peit, 2018). No obstante, producto de la democracia protegida promovida desde la transición (1990) y la herencia de la dictadura militar de Augusto Pinochet (1973-1989), la democracia en Chile se considera como limitada, elitista o en crisis (Delamaza, 2010; Garretón \& Garretón, 2010; Zarzuri, 2016). Según el PNUD (2019), el descontento hacia la democracia y la desconfianza hacia el sistema político, convive con un ethos republicano caracterizado por la alta valoración abstracta por la democracia, el voto, los partidos políticos y el estado de derecho. Para abordar la participación de los ciudadanos en un contexto como este, es menester considerar no solo su participación en elecciones dentro de la institucionalidad política, sino también otras formas de participación por fuera de la institucionalidad, pues ambas se encuentran a la base de los sistemas democráticos modernos (della Porta, 2013).

La conceptualización de la participación política es un tema de profundo debate en la literatura, producto de la expansión en las formas de participar (van Deth, 2014). Los bajos niveles de participación en elecciones y la desconfianza hacia las instituciones de representación, son factores que han forzado la reconceptualización y apertura a más formas de participación política (Ekman \& Amnå, 2012). En este sentido, actualmente la concepción de participación centrada en el voto conduce a un diagnóstico defectuoso, porque es ciega a la emergencia de otras formas de participación por fuera de la institucionalidad (Dalton, 2008, 2017; Dalton \& Klingemann, 2007). Por lo tanto, interesa estudiar la participación formal como el voto, junto con la participación activista que considera actividades extra institucionales como la protesta, a modo de capturar la participación tanto dentro como fuera de la institucionalidad (Miranda, 2018; Miranda et al., 2017). Ambas formas de participación son conceptualizadas como dimensiones de la participación cívica (Miranda et al., 2017), teniendo en común la toma de acción manifesta a influir en las decisiones políticas del gobierno o en los resultados políticos (Ekman \& Amnå, 2012).

Para el caso de Chile, múltiples investigaciones con un enfoque histórico-interpretativo, señalan que la actual participación formal y activista encuentra su configuración en la herencia del modelo socioeconómico y político-institucional de la dictadura, junto con la democracia protegida a cargo de los gobiernos de la Concertación (1990-2010). Ambas formas de participación cívica tienen cambios profundos desde el retorno a la democracia (1990), derivando en la actual paradoja del declive de la participación formal y el aumento de la participación activista.

Por un lado, el plebiscito de 1988 y las elecciones presidenciales de 1989, se caracterizaron por la alta inscripción y participación electoral (Contreras \& Navia, 2013; Navia, 2004; Toro, 2008), pero estos indicadores bajan sostenidamente en las últimas décadas (e.g., Bargsted, Somma \& Muñoz-Rojas, 2019; Contreras \& Navia, 2013; Corvalan \& Cox, 2015). Algunos investigadores señalan que el modelo político-institucional de la dictadura preservado por los gobiernos de la Concertación, produjo trabas y desincentivos para la participación formal, como sucedió con el sistema del registro electoral y el sistema binominal (Corvalan \& Cox, 2015; Garretón \& Garretón, 2010; PNUD, 2017a; Siavelis, 2016; Treviño et al., 2019). Una de las consecuencias más profundas, fue la falta de competencia y alternativas políticas (Lechner, 1994), lo que condujo en el largo plazo a una despartidización, crisis de representación y desafección de la sociedad civil con la política (Garretón, 2016; Luna, 2016; Siavelis, 2016; Somma, 2017). Actualmente los niveles de participación en elecciones son de los más bajos en América Latina y la desconfianza hacia las instituciones políticas toma un tono crítico (PNUD, 2017a, 2019), mientras que las élites políticas progresivamente se alejan o encapsulan más respecto de sus bases sociales (Castiglioni \& Rovira, 2016; Bargsted \& Maldonado, 2018; Luna, 2016).

Por otro lado, la década de 1980 se caracterizó por la cohesión entre partidos políticos y organizaciones sociales en torno al eje autoritarismo-democracia, lo que se tradujo en una serie de protestas en contra de la dictadura (Somma \& Bargsted, 2015; Somma \& Medel, 2017; Toro, 2008). No obstante, la década de 1990 con el retorno a la democracia, estuvo marcada por la ruptura de la relación entre los partidos y la ciudadanía organizada, a la falta de un enemigo común como la dictadura (Somma \& Bargsted, 2015; Somma \& Medel, 2017; Roberts, 2016). Las explicaciones para la consiguiente disminución de la participación activista, señalan los efectos nocivos de la atomización de la sociedad de mercado (Lechner, 1994; Somma \& Medel, 2017), la desarticulación de los actores sociales en dictadura y la desmovilización hacia abajo promovida por la Concertación, ante el peligro que significaba la acción colectiva para la fragilidad de la democracia (Castiglioni \& Rovira, 2016; Luna, 2016; Roberts, 2016; Somma \& Bargsted, 2015; Somma \& Medel, 2017). Sin embargo, las protestas aumentan desde la década del 2000 (Medel \& Somma, 2016; Somma \& Bargsted, 2015; Somma \& Medel, 2017), en particular con las movilizaciones estudiantiles del 2006, para luego extenderse a una diversidad de grupos que antes estaban desmovilizados (Medel \& Somma, 2016).

El panorama actual de la participación cívica en Chile, evidencia la paradoja del declive de la participación formal y el aumento de la participación activista. Los intentos del sistema político para minimizar las desigualdades en vías de labrar un consenso nacional y fomentar la eficiencia tecnocrática de la elite política (Roberts, 2016), hizo de las protestas la principal estrategia para que sean escuchadas las demandas ciudadanas (Medel \& Somma, 2016; Somma \& Medel, 2017), ante el debilitamiento de los canales institucionales de representación política y las dificultades las élites políticas para procesar el malestar (Garretón, 2016; Luna, 2016; Ruiz, 2015; Somma, 2017; Somma \& Bargsted, 2015; Somma \& Medel, 2017). De este modo, el descontento por la contradicción entre los principios del orden social de la dictadura y, los principios de un modelo socioeconómico justo y un orden político democrático (Garretón, 2016), se comienza a manifestar en la arena pública con una creciente movilización social de distintos actores que desafían e impugnan al sistema político, al margen de la institucionalidad (e.g., Bidegain, 2015; Donoso \& von Bülow, 2017; Somma, 2017; Somma \& Bargsted, 2015).

Este proceso político de la ciudadanía al margen e incluso en contra de la política institucional (PNUD, 2019), es entendido como una (re)politización de la sociedad chilena (Donoso \& von Bulow, 2017; PNUD, 2015, 2017a, 2019; Roberts, 2016), a pesar de coexistir con bajos niveles de identificación política, interés por la política y participación en elecciones (e.g., Bargsted et al., 2019; Núñez et al., 2018; PNUD, 2019). No obstante, este panorama es preocupante para la legitimidad de la democracia, no solo porque la baja participación formal afecta la representatividad de la diversidad de intereses y expectativas ciudadanas (PNUD, 2019), sino también por el déficit de una voz política igualitaria. En efecto, si otras formas de participación como la activista aumentan, sigue siendo problemático que solo unas voces se escuchen (Dalton, 2017). Una democracia sólida, inclusiva y participativa, debe tener como desafío superar la desigualdad política e incluir a todos los grupos sociales (PNUD, 2019).

Recientemente se han realizado investigaciones desde la ciencia política y la sociología, para explicar como afectan los factores que provocan las desigualdades de voz política en la participación de los ciudadanos (e.g., Bargsted et al., 2017; Dalton, 2017; Miranda, 2018; PNUD, 2017b; PNUD, 2019; Somma \& Bargsted, 2018; Somma, Bargsted \& Sánchez, 2020). La propuesta base es que, el grado de voz política para expresar intereses y demandas hacia el sistema político o en instancias de participación cívica (Dubrow, 2015 citado en PNUD, 2017b), son diferentes de acuerdo a las preferencias individuales y los factores sociodemográficos de los ciudadanos. La ciencia política y la sociología chilena sobre las preferencias individuales, aportan evidencia de que las personas con mayor participación formal (e.g., participación en elecciones), tienen una identificación política cargada hacia la derecha, participan más en organizaciones, confían más en instituciones políticas y tienen un mayor interés por la política (Bargsted et al., 2017; Castillo, Palacios, Joignant \& Tham, 2015b; Contreras \& Navia, 2013; Corvalan \& Cox, 2015; Núñez et al., 2018). Mientras tanto, las personas con mayor participación activista (e.g., participación en protestas), se identifican más con la izquierda, participan más en organizaciones, tiene un mayor interés por la política, se identifican más con las causas sociales y creen en la justicia redistributiva (Bargsted et al., 2017; Castillo et al., 2015b; Ortiz-Inostroza \& Lopez, 2017; Roberts, 2016).

Los factores sociodemográficos que explican la desigualdad de voz política, son principalmente investigados por la sociología. La literatura internacional en base al modelo de recursos (Verba, Schlozman \& Brady, 1995), el cual propone que los recursos económicos, educativos y estatus ocupacional inciden en una mayor propensión a la participación cívica, comprueba que quienes votan y asisten a protestas tienen un mayor estatus social (Brady et al., 1995; Dalton, 2017; Dalton, van Sickle \& Weldon, 2010; Dubrow, Slomczynski \& Tomescu-Dubrow, 2008; Lijphart, 1997; Marien, Hooghe \& Quintelier, 2010; Verba et al., 1995). En el plano nacional, se confirma que aquellas personas con niveles educativos más bajos tienen una menor propensión a la participación formal (Bargsted et al., 2017; Contreras \& Navia, 2013; Corvalan \& Cox, 2015), como también a la participación activista (Bargsted et al., 2017; Castillo et al., 2015b; Núñez et al., 2018; Ortiz-Inostroza \& Lopez, 2017; Roberts, 2016). Si bien existe evidencia de que sucede lo mismo con los niveles socioeconómicos más bajos (Bargsted et al., 2017; Contreras \& Navia, 2013; Corvalan \& Cox, 2015), algunas investigaciones señalan que en la participación activista tiene un efecto negativo (Roberts, 2016) o una relación no lineal (Ortiz-Inostroza \& Lopez, 2017). Esta divergencia puede ser porque el efecto positivo del nivel socioeconómico ocurre en protestas que buscan generar cambios en un aspecto particular de la sociedad, a diferencia de las protestas por la supervivencia material y social (Somma et al., 2020).

Siguiendo con los factores sociodemográficos de la desigualdad de voz política, existe un gran acuerdo de que edad es relevante para explicar la propensión a la participación formal y activista. No obstante, lo que sucede en este caso, es que la edad es un factor de desigualdad de voz política, dependiendo del tipo de participación cívica que se trate. De ese modo, la edad a la vez que es un factor de desigualdad política, configura en parte la paradoja de la participación cívica, expresada como el declive de la participación formal y el aumento de la participación activista. En lo sucesivo se precisa la preocupación por la participación cívica de los jóvenes, abordando las principales explicaciones y hallazgos, con el fin de abrir camino al estudio de la intención de participación cívica de los estudiantes.

\hypertarget{intenciuxf3n-de-participaciuxf3n-cuxedvica}{%
\section{Intención de participación cívica}\label{intenciuxf3n-de-participaciuxf3n-cuxedvica}}

La literatura señala que en el último tiempo disminuye la participación electoral y la membresía partidista de los jóvenes (definidos como personas entre 15 y 29 años), pero esto viene acompañado del aumento de su participación en otras formas de participación política, como es el caso de la participación activista (Dalton, 2006, 2017; Sandoval \& Carvallo, 2019; Zarzuri, 2016). Por lo tanto, los bajos niveles de participación formal de los jóvenes, más que significar una apatía política y despolitización generalizada, es un rechazo al sistema político actual, las instituciones políticas y los canales formales de participación (Arias-Cardona \& Alvarado, 2015; Instituto Nacional de la Juventud {[}INJUV{]}, 2017; Sandoval \& Carvallo, 2017; Zarzuri, 2016). La `nueva política' de los jóvenes desde lo cotidiano y `lo político' (Arias-Cardona \& Alvarado, 2015; Escobar, 2019; INJUV, 2017; Sandoval \& Carvallo, 2017, 2019; Zarzuri, 2016), asume una crítica a la concepción tradicional de democracia representativa, encapsulada en la participación formal y los márgenes institucionales de `la política' (INJUV, 2017; Escobar, 2019; Zarzuri, 2016).

Un cuerpo importante de investigaciones en Chile, a partir de fenómenos generacionales y sociopolíticos, han explicado la baja participación formal y el aumento de la participación activista de los jóvenes. Por un lado, las explicaciones a la baja participación formal parten en la década de 1990 (i.e., luego de la transición a la democracia) con la apatía política juvenil, caracterizada por el desencanto, la pérdida de interés en el mundo público y el refugio en el mundo privado (Navia, 2004; Parker, 2000). En esta línea, otras investigaciones argumentan el cambio generacional en la cultura política de los jóvenes, al desconfiar de la institucionalidad política, alejarse de las actividades colectivas y evitar la identificación ideológica (Carlin, 2006; Toro, 2008). Explicaciones más recientes desde un enfoque sociopolítico, señalan que el declive de la participación electoral de los jóvenes se debe a su baja inscripción en los padrones electorales luego del plebiscito de 1989 (Contreras \& Navia, 2013), derivando en el problema del reemplazo generacional, a raíz del sistema electoral heredado de la dictadura militar (Corvalan \& Cox, 2015). Asimismo, otros estudios aducen que los jóvenes prefieren otras formas de participación por fuera de la institucionalidad, ante la pérdida de la afectividad vinculante con la participación formal (Escobar, 2019; INJUV, 2017; Zarzuri, 2016).

Por otro lado, el aumento de la participación activista de los jóvenes, no quiere decir que sean los únicos protagonistas de la movilización social del último tiempo (Garretón, Joignant, Somma \& Campos, 2017), sino que son un actor social que crece y se fortalece por las movilizaciones estudiantiles en el período post autoritario (Sandoval \& Carvallo, 2019). En este sentido, la literatura coincide en que se trata de una generación `sin miedo' respecto a las generaciones anteriores que vivieron la dictadura militar, por lo cual, sienten la obligación, oportunidad y capacidad de transformar la sociedad heredada (Cárdenas, 2014; Cummings, 2015; Sandoval \& Carvallo, 2019). Puesto el legado dictatorial como un punto de referencia desde el cual los jóvenes se definen y actúan (Cummings, 2015), esta generación se caracteriza por estar socializada en el neoliberalismo, tocándole vivir la promesa incumplida del modelo de desarrollo, los costos de la modernización neoliberal y la desigualdad engendrada por la concentración económica (Ruiz \& Boccardo, 2014; Ruiz \& Caviedes, 2020a). De ese modo, `los hijos de la modernización neoliberal', encarnan los conflictos sociopolíticos producidos por la privatización de la vida cotidiana, la desprotección y exclusión social (Ruiz \& Caviedes, 2020b), derivando en un descontento puesto en la palestra pública por la brecha entre sus expectativas y la realidad (Cummings, 2015).

En este contexto de declive de la participación formal y aumento de la participación activista de los jóvenes, investigadores como Somma (2017) plantean que la edad produce una división del trabajo entre votantes y activistas, en donde los mayores se inclinan al voto y los jóvenes a la protesta. Como se dijo, la edad es un factor de desigualdad de voz política y actúa según la forma de participación cívica. Respecto a la participación formal, diversos estudios nacionales evidencian la menor propensión de los jóvenes a participar, siendo la más estudiada la participación electoral (Bargsted et al., 2017; Castillo et al., 2015b; Contreras, Morales \& Joignant, 2016; Contreras \& Navia, 2013; Corvalan \& Cox, 2015; Núñez et al., 2018; PNUD, 2017a; Toro, 2007, 2008). Si bien recientemente se prueba que las nuevas generaciones (nacidas entre 1991 y 1995) están votando más, porque vivieron las movilizaciones estudiantiles del 2006 y 2011 (Bargsted et al., 2019), la tendencia en el tiempo sobre su aumento aún no se comprueba. En cuanto a la participación activista, la literatura nacional muestra un gran acuerdo en que los jóvenes tienen una mayor propensión a participar, por lo general, estudiando la participación en protestas (Bargsted et al., 2017; Castillo et al., 2015b; Ortiz-Inostroza \& Lopez, 2017; Roberts, 2016).

Un resultado robusto en la literatura nacional, es que los factores sociodemográficos de la desigualdad de voz política, también se producen en la población joven del país. En base al mismo modelo de recursos estudiado en población adulta, la evidencia señala que aquellos jóvenes de menos estatus social (medido por los niveles socioeconómicos y educativos de las familias), tienen una menor propensión a la participación formal y activista (Contreras et al., 2016; Contreras \& Navia, 2013; Corvalan \& Cox, 2015; INJUV, 2017, 2019; Rozas \& Somma, 2020; Toro, 2007). Esto plantea desafíos para la legitimidad de la democracia en Chile, puesto que la desigualdad de voz política por estatus social se produce de manera transversal, con independencia del factor etario. No obstante, la participación cívica de los jóvenes es particularmente preocupante, porque sus intereses y expectativas ciudadanas se encuentran doblemente mal representadas por la democracia chilena. A los factores transversales del estatus social, se suma el factor etario de desigualdad de voz política, reflejado en la menor participación formal de los jóvenes.

Ante la preocupación por la baja participación cívica acumulada de los jóvenes, existe un importante cuerpo de investigación que se pregunta por como se puede promover su participación previamente, a partir del rol de formación ciudadana en las escuelas. La formación ciudadana en el Chile actual, es entendida como un proceso formativo continuo que busca promover no solo el conocimiento cívico, sino también las habilidades y actitudes necesarias para que los estudiantes se desenvuelvan como ciudadanos en una sociedad democrática (Ministerio de Educación {[}MINEDUC{]}, 2016). Asimismo, la literatura advierte que la formación ciudadana no se acota al currículum formal como mandato explícito hacia las escuelas, en tanto la cultura y el clima escolar pueden influir también en las actitudes, percepciones y creencias de los estudiantes como un currículum oculto (Apple, 1980; González, 2018). En tal sentido, la escuela es concebida como un espacio de socialización política (Cumsille \& Martínez, 2015; Deimel et al., 2020; Neundorf et al., 2016), con la posibilidad de promover la participación y solventar la desigualdad política (Bonhomme et al., 2015).

Diversas investigaciones en ciencias sociales se han interesado en probar si la formación ciudadana de las escuelas, como el aprendizaje formal de ciudadanía, el conocimiento cívico, un clima abierto en el aula, las habilidades ciudadanas y la percepción de influencia en las decisiones escolares (i.e., eficacia política en la escuela), promueve la intención de participación formal y activista de los estudiantes (Blaskó et al., 2019; Campbell, 2008; Castillo et al., 2015a; Deimel et al., 2020; Eckstein \& Noack, 2016; Hoskins et al., 2017; Isac et al., 2014; Kudrnáč, 2017; Manganelli et al.,, 2015; Miranda, 2018; Neundorf et al., 2016; Quintelier, 2010; Quintelier \& Hooghe, 2013; Treviño et al., 2019). Estas intenciones se reportan como la manera más cercana de medir y predecir la participación cívica a futuro, considerando que los estudiantes aún no tienen acceso a todas las formas de participación (Keating \& Janmaat, 2016; Quintelier \& Blais, 2016; Quintelier \& Hooghe, 2013). Por lo cual, se espera que al promover la intención de participación cívica de los estudiantes con la formación ciudadana de las escuelas, se obtengan impactos duraderos en el tiempo (Eckstein \& Noack, 2016; Keating \& Janmaat, 2016).

Si bien la socialización política de la escuela a través de la formación ciudadana es algo probado por la literatura, similar a como sucede con la participación cívica en adultos y jóvenes, el modelo de recursos comprueba que aquellos estudiantes de familias con menos estatus social, tienen una menor intención de participación formal y activista (Castillo et al., 2014; Castillo et al., 2015a; Kudrnáč, 2017; Manganelli et al., 2015; Miranda, 2018; Neundorf et al., 2016). En esta línea de investigación, ya no se conceptualiza la preocupación como una desigualdad de voz política, sino más bien, como la transmisión intergeneracional de la desigualdad política (Brady et al., 2015; Miranda, 2018; Schlozman et al., 2012). Este concepto quiere decir que las actitudes, creencias, conocimientos y comportamientos ciudadanos de los estudiantes, tienen una fuerte relación con el estatus social de sus familias (Miranda, 2018). De ese modo, la socialización política de la familia juega un rol importante en la intención de participación cívica de los estudiantes, en paralelo a la socialización política de las escuelas (Neundorf et al., 2016).

Ante la preocupación por la brecha generada por la transmisión intergeneracional de la desigualdad política, un creciente cuerpo de investigación se aboca a estudiar como es posible mitigar la brecha en la intención de participación cívica de los estudiantes, generadas por el estatus social de las familias (Campbell, 2008; Castillo et al., 2015a; Deimel, et al., 2020; Eckstein \& Noack, 2016; Gainous \& Martens, 2012; Hoskins et al., 2017; Kudrnáč, 2017; Neundorf et al., 2016). Este vínculo se mitiga cuando la formación ciudadana de las escuelas reduce la brecha por estatus social, pero la literatura sugiere que también puede ser potenciado cuando la formación ciudadana abre aún más la brecha (Campbell, 2008; Deimel et al., 2020; Hoskins et al., 2017). Es decir, mientras que el efecto mitigador se produce cuando los estudiantes desfavorecidos se benefician más de la formación ciudadana, por el contrario, el efecto potenciador es provocado cuando la formación ciudadana beneficia más a los estudiantes privilegiados (Campbell, 2008).

La evidencia sugiere mitigadores y advierte potenciadores del vínculo entre el estatus social familiar de los estudiantes y la intención de participación cívica, en su mayoría para el caso de la intención de participación formal. Uno de los más estudiados es el clima abierto en el aula, sobre el cual la literatura señala que mitiga la brecha por estatus social (Campbell, 2008; Castillo et al., 2015a; Eckstein \& Noack, 2016), mientras que otros estudios indican que la potencia (Gainous \& Martens, 2012) o incluso la deja intacta (Deimel et al., 2020; Hoskins et al., 2017; Kudrnáč, 2017). También la literatura prueba el rol de la participación en la escuela, la participación en actividades cívicas y la participación en decisiones escolares, encontrándose evidencia de que mitigan la brecha (Eckstein \& Noack, 2016; Gainous \& Marten, 2012), aunque otras investigaciones señalan que ni mitigan ni potencian el vínculo entre el estatus social familiar y la intención de participación (Deimel et al., 2020; Hoskins et al., 2017). A pesar de los resultados poco concluyentes, existe un consenso de que el aprendizaje formal de ciudadanía es un mitigador de la brecha por estatus social familiar, lo cual es probado por diversos estudios (Deimel et al., 2020; Gainous \& Marten, 2012, Hoskins et al., 2017; Neundorf et al., 2016).

Siguiendo a Deimel et al.~(2020), la evidencia sobre los mitigadores del vínculo entre el estatus social familiar y la intención de participación cívica de los estudiantes es inconsistente, no solo porque los resultados son poco concluyentes y la operacionalización de las variables difieren entre los estudios, sino también por las diferencias en el enfoque analítico adoptado. Mientras que algunos estudios examinan los efectos a nivel individual y escolar (Castillo et al., 2015a; Deimel et al., 2020; Campbell, 2008; Eckstein y Noack, 2016; Kudrnáč, 2017), otras investigaciones los analizan solo a nivel individual (Gainous \& Martens,2012; Hoskins et al., 2017; Neundorf et al., 2016).

\hypertarget{sentido-de-justicia-en-la-escuela}{%
\chapter{Sentido de justicia en la escuela}\label{sentido-de-justicia-en-la-escuela}}

Lo que se espera de las escuelas en las sociedades modernas, es que sean un punto de partida común capaz de corregir las desigualdades de herencia familiar, al brindar oportunidades educativas por igual a todos los estudiantes (Fdez. Enguita, 1990; Peña, 2015; Peña \& Toledo, 2017). Sin embargo, la literatura sociológica problematiza este rol, demostrando que las escuelas en muchos casos reproducen el origen en vez de corregirlo (Peña, 2015). Desde las Teorías de la Reproducción Social, la escuela es considerada como un aparato ideológico a cargo de legitimar y ocultar el orden estructural de clases, ante el peso del capital económico, cultural, social y simbólico heredado (Bourdieu \& Passeron, 1977). A pesar de esto, Freire (2014) plantea que si se desecha la concepción bancaria de la educación, la escuela puede ser una oportunidad para que los educandos problematicen y reflexionen sobre sus realidades, a modo de que los sometidos luchen por su emancipación. En este sentido, la sociología se pregunta por la justicia en la escuela, a modo de atender como la educación reproduce (o no) las estructuras de desigualdad (Sabbagh \& Resh, 2016).

Walzer (1983) conceptualiza la educación como una esfera de justicia autónoma de otras esferas de la sociedad, con sus propios principios y lógicas de funcionamiento. Un vasto cuerpo de investigaciones en ciencias sociales orienta empíricamente el estudio de la justicia en educación, teniendo como foco de atención las siguientes tres preguntas: 1) ¿qué principios y/o procedimientos se perciben como justos?; 2) ¿cuál es la magnitud percibida de la (in)justicia?; y 3) ¿cuáles son las consecuencias psicológicas y sociales de la (in)justicia? (Sabbagh \& Resh, 2016). En este estudio se presta atención a la última pregunta, la cual es contestada en referencia a distintas consecuencias que tiene la justicia en la escuela sobre los estudiantes. En tal sentido, la justicia en la escuela es considerada como parte del currículum oculto (Resh, 2010, 2018; Resh \& Sabbagh, 2013, 2014, 2017), el cual como se dijo, puede influir en las actitudes, percepciones y creencias de los estudiantes (Apple, 1980; González, 2018). Los estudios en este campo de investigación se acumulan en la psicología, la psicología social y la pedagogía, y en una menor cantidad se encuentra la literatura sociológica que colabora con la psicología social.

En cuanto a los aportes de la psicología, la evidencia muestra que la justicia en la escuela tiene efectos positivos sobre la motivación, el aprendizaje, la alegría de aprender y el bienestar académico (Berti, Molinari \& Speltini, 2010; Chory-Assad, 2002; Chory-Assad \& Paulsen, 2004; Ehrhardt-Madapathi, Pretsch \& Schmitt, 2018; Marcin, Morinaj \& Hascher, 2019; Peter \& Dalbert, 2010), pudiendo contribuir en las emociones positivas y la salud psicológica de los estudiantes (Mameli, Biolcati, Passini \& Mancini, 2018). A partir de diversos estudios en psicología social y pedagogía, la literatura señala que la justicia en la escuela tiene un efecto positivo sobre la confianza social por las personas y las instituciones formales, como también en la participación en la escuela y la comunidad (Lenzi et al., 2014; Resh, 2018; Resh \& Sabbagh, 2013, 2014). Asimismo, se demostró que promueve la identificación y el sentido de pertenencia a la escuela, fortalece el diálogo y confianza con los profesores, al mismo tiempo que reduce la deslegitimidad, el conflicto y las agresiones hacia ellos (Berti et al., 2010; Chory-Assad, 2002; Chory-Assad \& Paulsen, 2004; Čiuladienė \& Račelytė, 2016; Gouveia-Pereira, Vala \& Correia, 2017; Resh \& Sabbagh, 2013).

La contribución de la sociología en colaboraciòn con la psicología social, es un campo de investigación más reducido, pero existe una importante línea de investigación que se orienta al estudio del sentido de justicia en la escuela (Di Battista, Pivetti \& Berti, 2014; Gorard, 2011; Jasso \& Resh, 2002; Pretsch \& Ehrhardt-Madapathi, 2018; Resh, 2010, 2018; Resh \& Dalbert, 2007; Resh \& Sabbagh, 2009, 2013, 2014, 2016; 2017; Sabbagh \& Resh, 2016; Sabbagh et al., 2006). En este cuerpo de investigación, la evidencia empírica tiene en común el abordaje del efecto del género, estatus social de las familias, antecedentes de los estudiantes y tipo de escuela, sobre el sentido de justicia de los estudiantes (Gorard, 2011; Jasso \& Resh, 2002; Resh, 2010, 2018; Resh \& Dalbert, 2007; Resh \& Sabbagh, 2013, 2014, 2017). Un grupo de estudios en esta línea, recientemente se interesa cada vez más en investigar el efecto del sentido de justicia en la escuela sobre la socialización cívica (Resh \& Sabbagh, 2016), preocupándose tanto por el comportamiento como las actitudes cívicas de los estudiantes (Pretsch \& Ehrhardt-Madapathi, 2018; Resh, 2018; Resh \& Sabbagh, 2013, 2014, 2017). El presente estudio se enmarca en este cuerpo de literatura, con el objetivo de analizar el efecto del sentido de justicia en la escuela sobre la intención de participación cívica y probar si mitiga la brecha generada por el estatus social familiar de los estudiantes.

La evidencia hasta la fecha, aborda superficialmente el efecto de la justicia en la escuela sobre la intención de participación cívica, producto de una operacionalización que se limita a la percepción de los estudiantes sobre la justicia de los profesores (c.f. Eckstein \& Noack, 2016). En cambio, la literatura sobre el sentido de justicia en la escuela, brinda aportes conceptuales y empíricos que complejizan dicha operacionalización, al abrirse camino a la justicia distributiva y procesal en la escuela (Chory-Assad, 2002; Chory-Assad \& Paulsen, 2004; Čiuladienė \& Račelytė, 2016; Gouveia-Pereira et al., 2017; Jasso \& Resh, 2002; Resh, 2010, 2018; Resh \& Dalbert, 2007; Resh \& Sabbagh, 2009; 2013, 2014, 2017; Sabbagh \& Resh, 2016; Sabbagh et al., 2006), así como también en otros casos a la justicia interactiva e informativa (Di Battista et al., 2014; Čiuladienė \& Račelytė, 2016; Claus et al., 2012; Pretsch \& Ehrhardt-Madapathi, 2018).

Dicho lo anterior, en este estudio se consideran dos dimensiones del sentido de justicia en la escuela, las cuales son conceptualizadas como justicia distributiva y justicia relacional (e.g., Čiuladienė \& Račelytė, 2016; Claus et al., 2012; Resh, 2010; Resh \& Sabbagh, 2013; Sabbagh et al., 2006; Sabbagh \& Resh, 2016). En base a estas dimensiones, se sigue la línea de investigación que estudia el efecto del sentido de justicia en la escuela en la socialización cívica de los estudiantes (Resh \& Sabbagh, 2016), precisando en el efecto sobre la intención de participación formal y activista. Por lo tanto, en lo que sigue se profundiza sobre qué se entiende por cada una de las dimensiones del sentido de justicia en la escuela, para así abordar sus principales hallazgos, lo estudiado hasta ahora sobre su efecto en la intención de participación cívica de los estudiantes y el efecto mitigador sobre la brecha generada por el estatus social familiar.

\hypertarget{justicia-distributiva}{%
\section{Justicia distributiva}\label{justicia-distributiva}}

En el proceso de enseñanza-aprendizaje evaluar el rendimiento de los estudiantes es un momento importante y, por lo general, en la escuela se realiza mediante pruebas estandarizadas (Sabbagh \& Resh, 2016). Con estas pruebas se pretende dirimir entre el éxito y el fracaso en la escuela (Tateo, 2018), concibiendo a los estudiantes como individuos autónomos, competitivos y racionales; es decir, como responsables de su éxito y culpables de su fracaso (Dubet, 2011; Tarabini, 2015). Los estudiantes se evalúan de acuerdo a su posición en la distribución de notas, pero al estar regidas las pruebas estandarizadas por la distribución matemática de una Campana de Gauss, por necesidad unos pocos son definidos como sobresalientes, la mayoría como mediocres y otros pocos relegados a la noción de fracasados (Lampert, 2013). Tal forma de las pruebas estandarizadas, esconde que las notas sean una eufemización de clasificaciones sociales y, con resabios del darwinismo social, se sentencia que sólo los mejores son quienes triunfan (Kaplan \& Ferrero, 2003).

Una de las explicaciones de como se preserva la anterior situación, es por la aceptación e importancia que tiene el principio meritocrático en la escuela, en tanto la evidencia señala que la explicación preferida de los estudiantes para el éxito y fracaso, es por su propio esfuerzo y talento (Clycq, Ward Nouwen \& Vandenbroucke, 2014; Pansu, Dubois \& Dompnier, 2010). Los estudiantes internalizan las notas como valoraciones de sí mismos y explican su desempeño en la escuela por mérito (Lampert, 2013), eximiendo de responsabilidad a otros agentes sociales y a los factores estructurales o adscritos que los caracterizan (Fdez. Enguita, 1990; Peña \& Toledo, 2016; Tarabini, 2015). En este sentido, la meritocracia tiene expresión en la escuela con la distribución de notas, lo cual es denominado como justicia distributiva por la línea de investigación que trata el sentido de justicia en la escuela (Chory-Assad, 2002; Chory-Assad \& Paulsen, 2004; Jasso \& Resh, 2002; Pretsch \& Ehrhardt-Madapathi, 2018; Resh, 2010, 2018; Resh \& Dalbert, 2007; Resh \& Sabbagh, 2009, 2013, 2014, 2017; Sabbagh et al., 2006).

En la justicia distributiva, los recursos son el esfuerzo y el talento, mientras que las recompensas distribuidas son las notas en las pruebas (Resh, 2010; Resh \& Sabbagh, 2013; Sabbagh et al., 2006; Sabbagh \& Resh, 2016). Por consiguiente, el sentido de injusticia que emerge de la justicia distributiva, es provocada cuando la comparación entre la nota real y la nota esperada (o merecida), no coincide con las expectativas de los estudiantes (Resh, 2010, 2018; Resh \& Sabbagh, 2014, 2017). La literatura señala que la justicia distributiva tiene un carácter instrumental, porque una distribución justa (o injusta) de las notas, afecta la motivación de los estudiantes, sus posibilidades de éxito en la escuela y, en última instancia, sus futuras oportunidades en educación y en la vida (Resh, 2010; Resh \& Sabbagh, 2009). De ahí que la distribución de las notas sea un componente integral de la experiencia educativa de los estudiantes (Dalbert, 2004), puesto que además se trata de una recompensa que es valorada e importante para ellos (Resh, 2010; Resh \& Dalbert, 2007; Sabbagh \& Resh, 2016).

\hypertarget{efecto-a-nivel-individual}{%
\subsection{Efecto a nivel individual}\label{efecto-a-nivel-individual}}

La literatura sobre la justicia distributiva, señala que la escuela representa un `microcosmos' de la sociedad para los estudiantes, por lo que las experiencias de justicia (o injusticia) distributiva pueden transmitir nociones sobre la sociedad en general (Resh, 2018; Resh \& Sabbagh, 2014). Al ser parte del currículum oculto de las escuelas (Resh, 2010, 2018; Resh \& Sabbagh, 2013, 2014, 2017), la literatura clásica señala que la percepción justa (o injusta) de como se distribuyen las notas, puede moldear la visión de mundo y el `mapa social' que se construyen los estudiantes (Dreeben 1968; Deutsch 1979 citados en Resh, 2010). Es por esto que, la justicia distributiva en la escuela, transmite mensajes sobre la estructura y procesos sociales de la sociedad, los cuales son componentes básicos de una ciudadanía democrática (Connell, 1993 citado en Resh \& Sabbagh, 2014). En este sentido, la justicia distributiva puede ser parte integral de la socialización cívica de los estudiantes y un requisito para el desarrollo de la intención de participación cívica.

La evidencia empírica que trata la justicia distributiva en la escuela, recientemente investiga su efecto sobre la socialización cívica (Resh \& Sabbagh, 2016). Por un lado, respecto al comportamiento cívico se demostró que la justicia distributiva tiene un efecto negativo sobre la deshonestidad académica y la violencia escolar, pero no tuvo efectos significativos sobre la participación en actividades extracurriculares y la participación en voluntariado comunitario (Resh \& Sabbagh, 2017). Por otro lado, de acuerdo a las actitudes cívicas se comprueba el efecto positivo sobre la orientación democrática liberal (Resh \& Sabbagh, 2014), mientras que Pretsch y Ehrhardt-Madapathi (2018) evidencian un efecto positivo tanto en la aprobación a las derechos humanos y civiles, como en la aprobación a las instituciones democráticas. Si bien estos resultados indican que la justicia distributiva contribuye en las actitudes cívicas de los estudiantes, al mismo tiempo la literatura señala que no tiene efectos significativos sobre la confianza en las instituciones, ni tampoco en la confianza social (Resh \& Sabbagh, 2013, 2014). De manera más reciente, Resh (2018) advierte que estos efectos son diferentes de acuerdo al sector del sistema educativo (i.e., para el caso de Israel, depende de si son escuelas de judíos en general, judíos religiosos y árabes).

Hasta ahora no hay resultados concluyentes acerca de la importancia de la justicia distributiva sobre la socialización cívica de los estudiantes. Tampoco hay estudios dentro del cuerpo de literatura del sentido de justicia en la escuela, en que se analice el efecto de la justicia distributiva sobre la intención de participación cívica de los estudiantes. Sin embargo, debido a la evidencia que respalda la contribución de la justicia distributiva en la promoción de distintos comportamientos y actitudes cívicas deseables (Pretsch \& Ehrhardt-Madapathi, 2018; Resh y Sabbagh, 2014, 2017), junto con su imbricación con los componentes básicos de una ciudadanía democrática (Connell, 1993 citado en Resh \& Sabbagh, 2014), se espera un efecto positivo sobre la intención de participación formal y activista de los estudiantes. Tal nexo se apoya en que la justicia distributiva como parte del currículum oculto de las escuelas, puede cumplir una función latente en la configuración de las perspectivas sociales y el comportamiento de los estudiantes (Resh, 2010; Resh \& Sabbagh, 2009). La evidencia más cercana proviene de la población adulta en Chile, en que se comprueba que las preferencias por la redistribución son un factor relevante que promueve la participación en protestas (Castillo et al., 2015b).

\emph{H1a: la justicia distributiva a nivel individual tendrá un efecto positivo en la intención de participación formal y activista de los estudiantes.}

\hypertarget{efecto-a-nivel-escolar}{%
\subsection{Efecto a nivel escolar}\label{efecto-a-nivel-escolar}}

La literatura empírica que estudia la justicia distributiva en la escuela, tiene como común denominador considerarla en un nivel individual (c.f. Chory-Assad, 2002; Chory-Assad \& Paulsen, 2004; Čiuladienė \& Račelytė, 2016; Gorard, 2011; Gouveia-Pereira et al., 2017; Jasso \& Resh, 2002; Resh, 2010; Resh \& Dalbert, 2007), incluyendo la evidencia que analiza su efecto sobre la socialización cívica de los estudiantes (c.f. Pretsch \& Ehrhardt-Madapathi, 2018; Resh, 2018; Resh \& Sabbagh, 2013, 2014, 2017). Sin embargo, se propone estudiar también la justicia distributiva a nivel escolar, a modo de explorar entre escuelas el efecto sobre la intención de participación cívica de los estudiantes. Para lo cual, siguiendo la argumentación y la evidencia señalada a nivel individual, se espera que la justicia distributiva tenga un efecto positivo a nivel escolar sobre la intención de participación formal y activista de los estudiantes.

\emph{H1b: la justicia distributiva a nivel escolar tendrá un efecto positivo en la intención de participación formal y activista de los estudiantes.}

\hypertarget{justicia-relacional}{%
\section{Justicia relacional}\label{justicia-relacional}}

La escuela es un lugar importante en la vida cotidiana de los estudiantes, puesto que pasan gran parte del tiempo en ellas durante su infancia y adolescencia (Resh, 2018). Allí los estudiantes se relacionan diariamente con una diversidad de agentes, tales como compañeros, profesores, directivos, auxiliares, entre otros. No obstante, en particular los profesores son los agentes más importantes y con los cuales más se relacionan los estudiantes. No solo porque tienen la facultad de asignar un recurso valorado como las notas, sino también porque ofrecen elogios, apoyo, ayuda, aliento y estima (Resh, 2018; Resh \& Sabbagh, 2009, 2013, 2014, 2016, 2017; Sabbagh \& Resh, 2016; Sabbagh et al., 2006), lo cual es apreciado por los estudiantes. Por este motivo, tal como sucede con la justicia distributiva, la justicia que establecen los profesores en las relaciones con sus estudiantes, son un ámbito significativo de sus experiencias escolares (Dalbert, 2004).

Las relaciones entre profesores y estudiantes no son necesariamente justas, ya que pueden caracterizarse por la distribución desigual del poder que existe en la escuela (Mikula, Petri \& Tanzer, 1990 citado en Pretsch \& Ehrhardt-Madapathi, 2018). Por lo tanto, diversas investigaciones empíricas se interesan en abordar si los profesores mantienen una relación justa con sus estudiantes, lo que en una importante cantidad de casos se investiga como las relaciones profesor-estudiante en la escuela (e.g., Eckstein \& Noack, 2016; Isac et al., 2014; Quintelier \& Hooghe, 2013; Treviño et al., 2019). No obstante, el cuerpo de investigación que estudia empíricamente el sentido de justicia en la escuela, conceptualiza la percepción de los estudiantes sobre estas relaciones como justicia relacional (Di Battista et al., 2014; Čiuladienė \& Račelytė, 2016; Claus et al., 2012; Pretsch \& Ehrhardt-Madapathi, 2018; Resh, 2018; Resh \& Sabbagh, 2013, 2014, 2017). Si bien empíricamente ambas líneas de investigación son comparables porque refieren a la misma percepción de justicia en las relaciones de los profesores con sus estudiantes, conceptualmente tienen implicaciones distintas. Mientras que las relaciones profesor-estudiante en la escuela son parte del clima escolar, la justicia relacional es una dimensión del sentido de justicia en la escuela.
En la literatura que investiga la justicia relacional, no existe consenso sobre si es un aspecto de la justicia distributiva en la escuela, o bien si es un tipo de justicia que desborda lo distributivo. Por un lado, los estudios que consideran la justicia relacional como distributiva, argumentan que los profesores distribuyen el trato justo a sus estudiantes (e.g., Resh, 2018; Resh \& Sabbagh, 2013, 2014, 2017). Haciendo el símil con la justicia distributiva, plantean que la justicia relacional en vez de estar guiada por un principio de distribución meritocrático, se rige por un principio de distribución con énfasis en la igualdad (Resh \& Sabbagh, 2009, 2013; Sabbagh et al., 2006). Por consiguiente, los estudiantes son tratados como recursos y las recompensas son el trato justo de los estudiantes, en función del comportamiento igualitario de los profesores (Dalbert, 2004; Resh \& Sabbagh, 2017). Por otro lado, existen estudios que se refieren a la justicia relacional como una dimensión distinta a la justicia distributiva en la escuela (e.g., Di Battista et al., 2014; Čiuladienė \& Račelytė, 2016; Claus et al., 2012; Pretsch \& Ehrhardt-Madapathi, 2018). Estos estudios entienden la justicia relacional como la manera en que los profesores se relacionan con sus estudiantes, haciendo alusión a la importancia de que el trato justo en las relaciones interpersonales se base en el respeto, la dignidad y la cortesía (Čiuladienė \& Račelytė, 2016; Claus et al., 2012).

Es cuestionable entender la justicia relacional como distributiva, si los estudiantes son considerados como si fueran recursos y se limita la dimensión subjetiva que implican las relaciones justas entre profesores y estudiantes a una lógica economicista de recompensas (c.f. Dalbert, 2004; Resh \& Sabbagh, 2009, 2013, 2017; Sabbagh et al., 2006). Por lo cual, se conceptualiza la justicia relacional como parte de la justicia procesal en la escuela, la cual es definida por el mantenimiento de reglas formales e informales justas aceptadas por todos los estudiantes (Resh \& Sabbagh, 2014). En este caso, tal como lo plantea Di Battista et al.~(2014), una regla informal que es exigida por todos los estudiantes, es el trato justo interpersonal recibido por los profesores en la escuela.

\hypertarget{efecto-a-nivel-individual-1}{%
\subsection{Efecto a nivel individual}\label{efecto-a-nivel-individual-1}}

Así como la escuela es considerada un espacio de socialización política para los estudiantes (Cumsille \& Martínez, 2015; Deimel et al., 2020; Neundorf et al., 2016), existen investigaciones que conciben a los profesores como el principal agente a cargo de llevar a cabo esta socialización (e.g., Sampermans \& Claes, 2018). En este sentido, la literatura señala que de los profesores depende el aprendizaje cívico y las competencias cívicas de los estudiantes, sea comprometiéndose con prácticas colaborativas y promovimiento la reconciliación de conflictos (Isac et al., 2014), o si obran en base al respeto, la igualdad y la cortesía (Flanagan, Cumsille, Gill \& Gallay, 2007). Por consiguiente, cuando los profesores influyen en los estudiantes en el proceso de enseñanza, no sólo lo hacen a través de los contenidos de sus asignaturas, sino también mediante sus acciones y comportamientos en el aula (Sanderse, 2013 citado en Sampermans \& Claes, 2018). De acuerdo a esto, la justicia relacional puede contribuir en la socialización cívica, como también en particular en la intención de participación cívica de los estudiantes.

La evidencia empírica que trata la justicia relacional en la escuela, recientemente investiga su efecto sobre la socialización cívica (Resh \& Sabbagh, 2016). Respecto a los comportamientos cívicos, la evidencia señala que la justicia relacional permite al interior de la escuela disminuir la deshonestidad académica y la violencia escolar, junto con incentivar el voluntariado comunitario de los estudiantes (Resh \& Sabbagh, 2017). En cuanto a las actitudes cívicas, se comprueba que la justicia relacional tiene un efecto positivo en la orientación democrática liberal (Resh \& Sabbagh, 2014) y en la aprobación a los derechos humanos y civiles, como también en la aprobación a las instituciones democráticas (Pretsch \& Ehrhardt-Madapathi, 2018). Asimismo, es un acuerdo de que la justicia relacional promueve en los estudiantes la confianza social y la confianza en las instituciones formales (Resh \& Sabbagh, 2013, 2014), en todos los sectores del sistema educativo (i.e., para el caso de Israel, en las escuelas de judíos en general, judíos religiosos y árabes) (Resh, 2018).

La reciente evidencia sobre la justicia relacional muestra resultados alentadores sobre su contribución en la socialización cívica de los estudiantes. Es de esperarse que la justicia relacional también tenga un efecto positivo sobre la intención de participación cívica de los estudiantes, en cuanto los profesores además de impartir conocimiento, deben ser agentes que inculquen valores, actitudes y normas de comportamiento cívico para que sus estudiantes participen como ciudadanos comprometidos en el futuro (Resh, 2018). Al ser vistos los profesores como modelos a seguir (Sampermans \& Claes, 2018), el trato justo que establezcan en las relaciones con sus estudiantes, puede mejorar la aprobación y el compromiso de los estudiantes con las prácticas democráticas (Flanagan et al., 2007). En línea con tal argumento, Lenzi et al.~(2014) desde la psicología comunitaria, explican que el trato justo de los profesores fortalece el desarrollo de un sistema de creencias en los estudiantes en que se valora la igualdad, la equidad y los objetivos colectivos. Por consiguiente, plantean que tal sistema de creencias orientado al bien común, puede contribuir en las intenciones de participar de los estudiantes en el futuro.

Hasta la fecha la literatura empírica sobre el sentido de justicia en la escuela, aún no analiza el efecto de la justicia relacional sobre la intención de participación formal y activista de los estudiantes. No obstante, la evidencia acumulada sobre las relaciones profesor-estudiante en la escuela permite orientar tal efecto a nivel individual (e.g., Eckstein \& Noack, 2016; Quintelier \& Hooghe, 2013), pues como se dijo, esta línea de investigación es empíricamente comparable al cuerpo de literatura que estudia la justicia relacional.

La literatura internacional considerando múltiples países, indica que las relaciones profesor-estudiante en la escuela a nivel individual tienen un efecto positivo sobre la intención de participación política electoral, pero un efecto negativo en la intención de participación en protestas legales de los estudiantes (Quintelier \& Hooghe, 2013). Por su parte, una investigación en Alemania demuestra un efecto negativo a nivel individual de la justicia de los profesores sobre las intenciones de participación política (Eckstein \& Noack, 2016). En este estudio se considera la intención de participación cívica de los estudiantes como un misceláneo de diversas formas de participación, lo que se demuestra en un alfa de cronbach bastante bajo (\(\alpha\) = .55). Por lo tanto, en general la evidencia sobre las relaciones profesor-estudiante en la escuela muestra resultados poco concluyentes, pues las formas de medir la intención de participación cívica son distintas entre los estudios. Además, en la literatura revisada no se encuentra evidencia del efecto a nivel individual para el caso específico de Chile.

A pesar de la escasa evidencia y el problema que agrega que sean estudios con tipologías distintas, la literatura sobre las relaciones profesor-estudiante basadas en el trato justo, ayuda a orientar el sentido de la relación de la justicia relacional a nivel individual sobre la intención de participación formal y activista de los estudiantes. Ponderando los argumentos que señalan la contribución del trato justo de los profesores a la intención de participación cívica (Flanagan et al., 2007; Lenzi et al., 2014) y la evidencia empírica que brinda la literatura internacional, se espera que la justicia relacional tenga un efecto positivo a nivel individual sobre la intención de participación formal y activista de los estudiantes. En tal sentido, no hay suficiente evidencia empírica ni una explicación teórica para sostener un sentido de la relación distinto para la intención de participación activista de los estudiantes, tal como sugieren los resultados de Quintelier y Hooghe (2013).

\emph{H2a: la justicia relacional a nivel individual tendrá un efecto positivo en la intención de participación formal y activista de los estudiantes.}

\hypertarget{efecto-a-nivel-escolar-1}{%
\subsection{Efecto a nivel escolar}\label{efecto-a-nivel-escolar-1}}

Las investigaciones empíricas que estudian el efecto de la justicia relacional sobre la socialización cívica de los estudiantes, consideran esta dimensión del sentido de justicia en la escuela siempre a un nivel individual (c.f. Pretsch \& Ehrhardt-Madapathi, 2018; Resh, 2018; Resh \& Sabbagh, 2013, 2014, 2017). No obstante, la línea de investigación que estudia el trato justo en las relaciones profesor-estudiante en paralelo a la justicia relacional, por lo general concibe estas relaciones como parte del clima escolar (e.g., Eckstein \& Noack, 2016; Isac et al., 2014; Quintelier \& Hooghe, 2013). Por lo cual, por antonomasia se infiere que las relaciones justas que establecen los profesores, son comprendidas por estas investigaciones como una percepción general que realizan los estudiantes sobre su escuela. Dicho de otro modo, la justicia relacional puede ser considerada una evaluación general hacia los profesores en la escuela, a pesar de que sea reportado a nivel individual por los estudiantes.

La literatura sobre el sentido de justicia en la escuela, hasta ahora no ha estudiado el efecto de la justicia relacional a nivel escolar sobre la intención de participación formal y activista de los estudiantes, pero existe una vasta evidencia en paralelo sobre el trato justo en las relaciones profesor-estudiante. Por un lado, la literatura internacional considerando una cantidad importante de países, prueba un efecto positivo a nivel escolar de las relaciones profesor-estudiante en la escuela sobre la ciudadanía convencional (Isac et al., 2014), aunque otro estudio de la misma categoría no encontró un efecto positivo sobre las intenciones de participación electoral de los estudiantes (Quintelier \& Hooghe, 2013). Por otro lado, la anterior literatura internacional mantiene un consenso sobre el efecto positivo a nivel escolar de las relaciones profesor-estudiantes sobre la ciudadanía relacionada con los movimientos sociales (Isac et al., 2014) o la intención de participar en protestas legales (Quintelier \& Hooghe, 2013). La evidencia para el caso de Chile, corrobora el efecto positivo a nivel escolar de las relaciones profesor-estudiantes en las expectativas de participación electoral y las expectativas de participar en protestas legales (Treviño et al., 2019).

Si bien se mantiene el problema de que la evidencia conceptualiza de diversas maneras la intención de participación cívica de los estudiantes, la literatura sobre las relaciones profesor-estudiante en la escuela permite dirimir el sentido de la relación a nivel escolar de la justicia relacional. La evidencia es suficiente para sostener que la justicia relacional a nivel escolar, tendrá un efecto positivo sobre la intención de participación formal y activista de los estudiantes. Aquello se encuentra en coincidencia con la explicación desarrollada por Flanagan et al.~(2007) y Lenzi et al.~(2014) para tal efecto a nivel individual, junto con la evidencia empírica reciente sobre el caso de Chile por Treviño et al.~(2019).

\emph{H2b: la justicia relacional a nivel escolar tendrá un efecto positivo en la intención de participación formal y activista de los estudiantes.}

\hypertarget{efecto-mitigador-de-la-justicia-distributiva-y-relacional}{%
\section{Efecto mitigador de la justicia distributiva y relacional}\label{efecto-mitigador-de-la-justicia-distributiva-y-relacional}}

La literatura internacional y nacional es consistente al probar con el modelo de recursos, la brecha generada por el estatus social de las familias en la intención de participación cívica de los estudiantes (Castillo et al., 2014; Castillo et al., 2015a; Kudrnáč, 2017; Manganelli et al., 2015; Miranda, 2018; Neundorf et al., 2016). Esto que es conceptualizado como la transmisión intergeneracional de la desigualdad política (Brady et al., 2015; Miranda, 2018; Schlozman et al., 2012), para una creciente área de investigación es un desafío y preocupación identificar como mitigar la brecha (Campbell, 2008; Castillo et al., 2015a; Deimel, et al., 2020; Eckstein \& Noack, 2016; Gainous \& Martens, 2012; Hoskins et al., 2017; Kudrnáč, 2017; Neundorf et al., 2016). En este sentido, interesa probar si la justicia distributiva y relacional en la escuela, logran mitigar el vínculo entre el estatus social familiar y la intención de participación formal y activista de los estudiantes. De acuerdo a lo planteado por Deimel et al.~(2020), el efecto mitigador de ambas dimensiones del sentido de justicia en la escuela solo será analizado a nivel individual.

La literatura empírica que investiga la justicia distributiva, hasta ahora no ha probado si mitiga la brecha generada generada por el estatus social familiar, pero existe evidencia complementaria que alienta su estudio. La evidencia disponible señala que los niveles de justicia distributiva en la escuela, no están significativamente afectados por el estatus social de sus familias (Resh, 2010; Resh \& Dalbert, 2007). Esto es un aspecto a favor si logra mitigar el vínculo entre el estatus social familiar y la intención de participación cívica de los estudiantes, puesto que otros mitigadores ya probados sobre este vínculo, tales como un clima abierto en el aula, la participación en actividades cívicas y la educación formal en ciudadanía, se caracterizan por su desigualdad de acceso (Deimel et al., 2020; Hoskins et al., 2017). Este acceso desigualdad refiere a las barreras que obstaculizan a los estudiantes más desfavorecidos beneficiarse de la formación ciudadana, exacerbando la desigualdad al proporcionar más preparación a aquellos que ya es probable que alcancen una cantidad suficiente de voz política (Kahne \& Middaugh, 2008 citado en Deimel et al., 2020). Al carecer la justicia distributiva de esta desigualdad de acceso, puede ser valioso su aporte como parte del currículum oculto en la escuela, para mitigar a nivel individual el vínculo entre el estatus social familiar y la intención de participación formal y activista de los estudiantes.

\emph{H3a: la justicia distributiva a nivel individual logrará mitigar el vínculo entre el estatus social familiar y la intención de participación formal y activista de los estudiantes.}

Para el caso de la justicia relacional, la literatura empírica tampoco investiga si mitiga el vínculo entre el estatus social de las familias y la intención de participación cívica. No obstante, en paralelo la literatura sobre las relaciones profesor-estudiante basadas en el trato justo, muestra resultados alentadores sobre el efecto mitigador que pueda tener la justicia relacional. Al respecto, las relaciones profesor-estudiante en la escuela en múltiples países de Europa, han logrado mitigar la brecha de género en las actitudes hacia la igualdad de género (Sampermans \& Claes, 2018), como también la brecha por estatus social en el caso de la participación social de los estudiantes (Wanders, Dijkstra, Maslowski \& van der Veen, 2020; Wanders, van der Veen, Dijkstra \& Maslowski, 2020). La evidencia internacional más cercana la brindan Eckstein y Noack (2016), al probar que la justicia de los profesores en la escuela mitiga la brecha generada por el estatus social de las familias, en la intención de participación política de los estudiantes. A pesar de la limitación de que este estudio no diferencia entre la intención de participación formal y activista, junto con la falta de evidencia para el caso de Chile, es de esperar que la justicia relacional logre mitigar a nivel individual la brecha generada por el estatus social familiar.

\emph{H3b: la justicia relacional a nivel individual logrará mitigar el vínculo entre el estatus social familiar y la intención de participación formal y activista de los estudiantes.}

\hypertarget{datos-medidas-y-muxe9todo-de-anuxe1lisis}{%
\chapter{Datos, medidas y método de análisis}\label{datos-medidas-y-muxe9todo-de-anuxe1lisis}}

\hypertarget{base-de-datos-agencia-de-la-educaciuxf3n-2017}{%
\section{Base de datos Agencia de la Educación (2017)}\label{base-de-datos-agencia-de-la-educaciuxf3n-2017}}

En este estudio se ocupan datos secundarios, obtenidos de la encuesta desarrollada por la Agencia de Calidad de la Educación en el año 2017. La encuesta tiene por objetivo realizar una evaluación sobre la formación ciudadana en Chile, a una muestra representativa de estudiantes de 8° básico. La evaluación cuenta con una muestra de escuelas con un total de 10.213 estudiantes de 8° básico encuestados, incluyendo cuestionarios actitudinales tanto para los estudiantes como para sus padres. Para el caso de esta investigación, se cuenta con una muestra de 7072 estudiantes (49\% son hombres y 51\% son mujeres).

La Base de datos Agencia de la Educación (2017), es la evaluación más actual sobre formación ciudadana en Chile, por sobre los datos nacionales del Estudio Internacional de Educación Cívica y Formación Ciudadana (ICCS) del año 2016. Ambas bases de datos cuentan con escalas e indicadores muy similares, lo que brinda respaldo de fiabilidad y validez a las mediciones, las cuales han sido probadas por el estudio ICCS en su versión del 2009 y 2016. En este sentido, la Base de datos Agencia de la Educación (2017) permite obtener resultados comparables al estudio ICCS, aunque también agrega mediciones nuevas.
\newpage

\hypertarget{medidas}{%
\section{Medidas}\label{medidas}}

\hypertarget{variables-dependientes-intenciuxf3n-de-participaciuxf3n-cuxedvica}{%
\subsection{Variables dependientes: intención de participación cívica}\label{variables-dependientes-intenciuxf3n-de-participaciuxf3n-cuxedvica}}

Según Miranda (2018), hay una diversidad de nombres para referirse a la misma forma de participación de los estudiantes, lo que genera una confusión conceptual y dificultades para realizar comparaciones entre estudios (c.f. Isac et al., 2014; Manganelli et al., 2015; Quintelier \& Hooghe, 2013; Treviño et al., 2019). Por tal motivo, Miranda et al.~(2017) ponen a prueba empírica un modelo conceptual de participación ciudadana juvenil, mediante un análisis factorial confirmatorio multigrupo y equivalencia de medición para estudiantes de octavo grado en 38 países (n = 139.875) que participaron en el estudio ICCS del año 2009. Entre sus resultados, obtienen una dimensión de participación cívica intencional e informada, la cual incluye la participación formal y activista de los estudiantes. En ambas formas de participación cívica, encontraron que los indicadores propuestos miden los constructos de manera aceptable.

En este estudio se trabaja con los ítems ya validados para medir los constructos latentes de participación formal y activista de acuerdo a Miranda et al.~(2017), en particular los que refieren a la intención de participación de los estudiantes. Para la muestra con la que se trabaja, la escala que mide la intención de participación formal mostró una excelente fiabilidad (\(\alpha\) = .92), mientras que la escala sobre la intención de participación activista mostró una consistencia interna apenas aceptable (\(\alpha\) = .65). Los ítems de cada escala y las categorías de respuesta se resumen en la Tabla 1.

Tabla 1

\emph{Indicadores para medir la intención de participación cívica}

\begin{longtable}[]{@{}lll@{}}
\toprule
\begin{minipage}[b]{0.28\columnwidth}\raggedright
\textbf{Participación Formal}\strut
\end{minipage} & \begin{minipage}[b]{0.32\columnwidth}\raggedright
\textbf{Participación activista}\strut
\end{minipage} & \begin{minipage}[b]{0.31\columnwidth}\raggedright
\textbf{Categorías de respuesta}\strut
\end{minipage}\tabularnewline
\midrule
\endhead
\begin{minipage}[t]{0.28\columnwidth}\raggedright
Votar en elecciones municipales (alcalde y concejales)\strut
\end{minipage} & \begin{minipage}[t]{0.32\columnwidth}\raggedright
Participar escribiendo artículos para un diario o sitio web escolar\strut
\end{minipage} & \begin{minipage}[t]{0.31\columnwidth}\raggedright
Seguro que no haría eso\strut
\end{minipage}\tabularnewline
\begin{minipage}[t]{0.28\columnwidth}\raggedright
Votar en elecciones presidenciales\strut
\end{minipage} & \begin{minipage}[t]{0.32\columnwidth}\raggedright
Contactando a un alcalde, senador o diputado\strut
\end{minipage} & \begin{minipage}[t]{0.31\columnwidth}\raggedright
Probablemente no haría eso\strut
\end{minipage}\tabularnewline
\begin{minipage}[t]{0.28\columnwidth}\raggedright
Informarte sobre los candidatos antes de votar en una elección\strut
\end{minipage} & \begin{minipage}[t]{0.32\columnwidth}\raggedright
Participando en una marcha o manifestaciones pacífica\strut
\end{minipage} & \begin{minipage}[t]{0.31\columnwidth}\raggedright
Probablemente haría eso\strut
\end{minipage}\tabularnewline
\begin{minipage}[t]{0.28\columnwidth}\raggedright
\strut
\end{minipage} & \begin{minipage}[t]{0.32\columnwidth}\raggedright
\strut
\end{minipage} & \begin{minipage}[t]{0.31\columnwidth}\raggedright
Seguro que haría eso\strut
\end{minipage}\tabularnewline
\bottomrule
\end{longtable}

\emph{Fuente: Elaboración propia en base a las preguntas de la Agencia de la Educación (2017).}

\hypertarget{variables-independientes-sentido-de-justicia-en-la-escuela}{%
\subsection{Variables independientes: sentido de justicia en la escuela}\label{variables-independientes-sentido-de-justicia-en-la-escuela}}

\textbf{Justicia distributiva}

La justicia distributiva como dimensión del sentido de justicia en la escuela, es medida en diversas ocasiones de forma cuantitativa (e.g., Chory-Assad, 2002; Chory-Assad \& Paulsen, 2004; Jasso \& Resh, 2002; Pretsch \& Ehrhardt-Madapathi, 2018; Resh, 2010, 2018; Resh \& Dalbert, 2007; Resh \& Sabbagh, 2013, 2014, 2017) y se enfatiza en el principio meritocrático que debe regir la distribución de las notas (Resh \& Sabbagh, 2009, 2013; Sabbagh et al., 2006). En este estudio el sentido de justicia sobre la distribución meritocrática de las notas, se mide a nivel individual con una escala de 2 indicadores reportados por los estudiantes (``La inteligencia es importante para obtener buenas notas'' y ``El esfuerzo es importante para obtener buenas notas'') y a nivel escolar se crea una variable ficticia con el promedio de la escala por escuela. Cada indicador tiene 4 puntos medidos en una escala Likert, donde 1 es ``Muy en desacuerdo'' y 4 es ``Muy de acuerdo''. La correlación entre los 2 indicadores en la muestra es r = .19.

\textbf{Justicia relacional}

La justicia relacional como dimensión del sentido de justicia en la escuela, la cual hace alusión al trato justo de los profesores en las relaciones que establecen con sus estudiantes en la escuela, es una medición probada por múltiples investigaciones (e.g., Isac et al., 2014; Quintelier \& Hooghe, 2013; Treviño et al., 2019). En esta investigación el sentido de justicia sobre las relaciones interpersonales entre profesores y estudiantes, se miden a nivel individual con una escala de 5 indicadores reportados por los estudiantes (``La mayoría de mis profesores me tratan de manera justa'', ``Los estudiantes se llevan bien con la mayoría de los profesores'', ``A la mayoría de los profesores les interesa el bienestar de los estudiantes'', ``La mayoría de mis profesores realmente escuchan lo que tengo que decir'' y ``Si necesito ayuda extra, mis profesores me la darán'') y a nivel escolar se crea una variable ficticia con el promedio de la escala por escuela. Los indicadores tienen 4 puntos y están medidos en una escala Likert, siendo 1 ``Muy en desacuerdo'' y 4 ``Muy de acuerdo''. La medición de la justicia relacional mostró una buena fiabilidad en la muestra (\(\alpha\) = .83).

\hypertarget{variables-control}{%
\subsection{Variables control}\label{variables-control}}

\textbf{Género}

El género del estudiante es una variable dicotómica a nivel individual, que puede tomar como valor 0 si es hombre y 1 si es mujer.
\newpage

\textbf{Interés por temas políticos y sociales}

El interés de los estudiantes por temas políticos y sociales, es un variable ordinal a nivel individual medida como escala Likert de 4 puntos, siendo 1 ``Nada'' y 4 ``Mucho''.

\textbf{Estatus social}

Ante las dificultades de preguntarle a un adolescente de 8° básico el nivel de ingresos de su familia, múltiples estudios señalan que la cantidad de libros en el hogar y el logro educativo son variables proxy del estatus social familiar de los estudiantes (e.g., Campbell, 2008; Castillo et al., 2014; Castillo et al., 2015a; Kudrnáč, 2017; Miranda, 2018; Quintelier, 2010; Quintelier \& Hooghe, 2013; Resh, 2018; Resh \& Sabbagh, 2014; Treviño et al., 2019). En primer lugar, la cantidad de libros es una variable reportada por el estudiante con un nivel de medición ordinal de 5 puntos, siendo 1 ``Pocos o muy pocos (0 a 10 libros)'' y 5 ``Suficientes para llenar tres o más libros (más de 200 libros)''. En segundo lugar, el logro educativo es medido como la expectativa de educación del estudiante, cuyo nivel de medición de la variable es ordinal con 4 puntos, donde 1 es ``8° básico'' y 4 es ``Universidad''. Para controlar a nivel escolar, se crean variables ficticias con el promedio de las medidas por escuela.

\hypertarget{muxe9todo-para-el-anuxe1lisis-de-datos}{%
\section{Método para el análisis de datos}\label{muxe9todo-para-el-anuxe1lisis-de-datos}}

La estructura jerárquica de las hipótesis y los datos a utilizar (i.e., estudiantes anidados dentro de escuelas), vuelven pertinente el análisis de datos con modelos lineales jerárquicos o modelos multinivel (Hox, 2010). Este es un método apropiado para datos de dos o más niveles de medición anidados, permitiendo la descomposición y estimación simultánea de efectos a nivel individual y niveles agregados (ibíd.). La estrategia metodológica, por lo tanto, permite estimar la varianza de las variables dependientes a través de los niveles de agregación, así como la proporción de varianza explicada en cada nivel. El software utilizado para este análisis multinivel es R, a través del paquete `lme4' desarrollado por Douglas Bates, al proporcionar una sintaxis más concisa y más flexible respecto a anteriores librerías (Finch, Bolin \& Kelley, 2014).

Los análisis de la intención de participación cívica basados en modelos multinivel, siguen la siguiente fórmula:

\(\text{y}_{ij}=\gamma{00}+\gamma{10}\text{X}_j+\gamma_{01}\text{Z}j+\mu_{oj}+\text{r}_{ij}\)

Donde \(\text{y}_{ij}\) es el valor de la variable dependiente, es decir, los niveles de intención de participación cívica de los estudiantes (un modelo con la intención de participación formal y otro con la intención de participación activista); \(\gamma{00}\) es el gran intercepto o promedio general estimado a partir de los resultados de todos los estudiantes; \(\gamma{10}\text{X}_j\) es el valor de la pendiente asociado a un efecto fijo de variable individual (el género, interés por temas políticos y sociales, variables de estatus social y las dimensiones del sentido de justicia a nivel individual); \(\gamma_{01}\text{Z}j\) es el valor de la pendiente asociado a un efecto fijo de variable contextual (variables ficticias tanto del estatus social como las dimensiones del sentido de justicia a nivel escolar); \(\mu_{oj}\) es el efecto aleatorio para el intercepto según anidación, en este caso, según escuelas; y \(\text{r}_{ij}\) son los residuos a nivel individual.

Los modelos multinivel han sido utilizados por diversos estudios que analizan el efecto de la socialización política de la escuela sobre la intención de participación cívica (Campbell, 2008; Castillo et al., 2015a; Deimel et al., 2020; Eckstein \& Noack, 2016; Isac et al., 2014; Kudrnáč, 2017; Manganelli et al., 2015; Quintelier, 2010; Quintelier \& Hooghe, 2013; Treviño et al., 2019), incluyendo investigaciones que prueban el efecto del sentido de justicia en la escuela y la socialización cívica de los estudiantes (Resh \& Sabbagh, 2013, 2014, 2017). Así mismo, investigaciones que estudian como mitigar el vínculo entre el estatus social familiar y la intención de participación cívica de los estudiantes, también han utilizado modelos multinivel (Castillo et al., 2015a; Deimel et al., 2020; Campbell, 2008; Eckstein y Noack, 2016; Kudrnáč, 2017).

\hypertarget{bibliografuxeda}{%
\chapter*{Bibliografía}\label{bibliografuxeda}}
\addcontentsline{toc}{chapter}{Bibliografía}

Apple, M. (1980). The other side of the hidden curriculum: correspondence theories and the labor process. The Journal of Education, 162(1), 47--66. Recuperado de \url{http://www.jstor.org/stable/42741975}

Arias-Cardona, A. M. \& Alvarado, S. V. (2015). Jóvenes y política: de la participación formal a la movilización informal. Revista Latinoamericana de Ciencias Sociales, Niñez y Juventud, 13(2), 581--594. doi: 10.11600/1692715x.1322241014

Bargsted, M. \& Maldonado, L. (2018). Party identification in an encapsulated party system: The case of postauthoritarian Chile. Journal of Politics in Latin America, 10(1), 29--68. doi: \url{https://doi.org/10.1177/1866802X1801000102}

Bargsted, M., Somma, N., Campos, T. \& Joignant, A. (2017). Ciudadanía y democracia: desigualdades de voz política. Notas COES de Política Pública, 11, 1--17. doi: 10.13140 / RG.2.2.28798.28480

Bargsted, M., Somma, N. \& Muñoz-Rojas, B. (2019). Participación electoral en Chile. Una aproximación de edad, período y cohorte. Revista de ciencia política, 39(1), 75--98. doi: \url{http://dx.doi.org/10.4067/S0718-090X2019000100075}

Berti, C., Molinari, L. \& Speltini, G. (2010). Classroom justice and psychological engagement: students' and teachers' representations. Social Psychology of Education, 13(4), 541--556. doi: \url{https://doi.org/10.1007/s11218-010-9128-9}

Bidegain, G. (2015). Autonomización de los movimientos sociales e intensificación de la protesta: estudiantes y mapuches en Chile (1990--2013) (Tesis de doctorado). Pontificia Universidad Católica de Chile, Santiago de Chile.

Blaskó, Z., Dinis da Costa, P. \& Vera-Toscano, E. (2019). Non-cognitive civic outcomes: How can education contribute? European evidence from the ICCS 2016 study. International Journal of Educational Research, 98, 366--378. doi: \url{https://doi.org/10.1016/j.ijer.2019.07.005}

Bonhomme, M., Cox, C., Tham, M. \& Lira, R. (2015). La educación ciudadana escolar de Chile `en acto': prácticas docentes y expectativas de participación política de estudiantes. En C. Cox \& J. C. Castillo (Eds.), Aprendizaje de la ciudadanía: Contextos, experiencias y resultados (pp.~373--425). Santiago de Chile: Ediciones Universidad Católica de Chile.

Bourdieu, P. \& Passeron, J. C. (1977). La reproducción. Elementos para una teoría del sistema de enseñanza. Barcelona: Laia.

Brady, H. E., Schlozman, K. L. \& Verba, S. (2015). Political Mobility and Political Reproduction from Generation to Generation. The ANNALS of the American Academy of Political and Social Science, 657(1), 149--173. doi: \url{https://doi.org/10.1177/0002716214550587}

Brady H, E., Verba, S. \& Schlozman K. L. (1995). Beyond SES: a resource model of political participation. American Political Science Review, 89(2), 271--294. doi: 10.2307/2082425

Campbell, D. E. (2008). Voice in the Classroom: How an Open Classroom Climate Fosters Political Engagement Among Adolescents. Political Behavior, 30, 437--454. doi: \url{https://doi.org/10.1007/s11109-008-9063-z}

Carlin, R. (2006). The decline of citizen participation in electoral politics in post-authoritarian Chile. Democratization, 13(4), 632--651. doi: \url{https://doi.org/10.1080/13510340600791921}

Castiglioni, R. \& Rovira, C. (2016). Challenges to political representation in contemporary Chile. Journal of Politics in Latin America, 8(3), 3--24. doi: \url{https://doi.org/10.1177/1866802X1600800301}

Castillo, J. C., Miranda, D., Bonhomme, M., Cox, C. \& Bascopé, M. (2014). Social inequality and changes in students' expected political participation in Chile. Education, Citizenship and Social Justice, 9(2), 140--156. doi: \url{https://doi.org/10.1177/1746197914520650}

Castillo, J. C., Miranda, D., Bonhomme, M., Cox, C. \& Bascopé, M. (2015a). Mitigating the political participation gap from the school: the roles of civic knowledge and classroom climate. Journal of Youth Studies, 18(1), 16--35. doi: \url{https://doi.org/10.1080/13676261.2014.933199}

Castillo, J. C., Palacios, D. Joignant, A. \& Tham, M. (2015b). Inequality, Distributive Justice and Political Participation: An Analysis of the Case of Chile. Bulletin of Latin American Research, 34(4), 486--502. doi: \url{https://doi.org/10.1111/blar.12369}

Chory-Assad, R. (2002). Classroom justice: Perceptions of fairness as a predictor of student motivation, learning, and aggression. Communication Quarterly, 50(1), 58--77. doi: \url{https://doi.org/10.1080/01463370209385646}

Chory‐Assad, R. \& Paulsel, M. (2004) Classroom justice: student aggression and resistance as reactions to perceived unfairness. Communication Education, 53(3), 253--273. doi: \url{https://doi.org/10.1080/0363452042000265189}

Čiuladienė, G. \& Račelytė, D. (2016). Perceived unfairness in teacher-student conflict situations: students' point of view. Polish Journal of Applied Psychology, 14(1), 49--66. doi: 10.1515/pjap-2015-0049

Claus, C., Chory, R. \& Malachowski, C. (2012). Student Antisocial Compliance-Gaining as a Function of Instructor Aggressive Communication and Classroom Justice. Communication Education, 61(1), 17--43. doi: \url{https://doi.org/10.1080/03634523.2011.619270}

Clycq, N., Ward Nouwen, M. A. \& Vandenbroucke, A. (2014). Meritocracy, deficit thinking and the invisibility of the system: Discourses on educational success and failure. British Educational Research Journal, 40(5), 796--819. doi: \url{https://doi.org/10.1002/berj.3109}

Contreras, G., Morales, M. \& Joignant, A. (2016). The Return of Censitary Suffrage? The Effects of Automatic Voter Registration and Voluntary Voting in Chile. Democratization, 23(3), 520--544. doi: \url{https://doi.org/10.1080/13510347.2014.986720}

Contreras, G. \& Navia, P. (2013). Diferencias generacionales en la participación electoral en Chile, 1988-2010. Revista de ciencia política, 33(2), 419--441. doi: \url{http://dx.doi.org/10.4067/S0718-090X2013000200001}

Corvalán, A. \& Cox, P. (2015). Participación y desigualdad electoral en Chile. En C. Cox \& J. C. Castillo (Eds.), Aprendizaje de la ciudadanía: Contextos, experiencias y resultados (pp.~175--206). Santiago de Chile: Ediciones Universidad Católica de Chile.

Cummings, P. (2015). Democracy and Student Discontent: Chilean Student Protest in the Post-Pinochet Era. Journal of Politics in Latin America, 7(3), 49--84. doi: \url{https://doi.org/10.1177/1866802X1500700302}

Cumsille, P. \& Martínez, M. L. (2015). La escuela como contexto de socialización política: Influencias colectivas e individuales. En C. Cox \& J. C. Castillo (Eds.), Aprendizaje de la Ciudadanía: Contextos, Experiencias y Aprendizajes (pp.~431--457). Santiago de Chile: Ediciones Universidad Católica de Chile.

Dalbert, C. (2004). The implications and functions of just and unjust experiences in school. En C. Dalbert \& H. Sallay (Eds.), The justice motive in adolescence and young adulthood (pp.~117--134). New York: Routledge. doi: \url{https://doi.org/10.4324/9780203575802}

Dalton, R. (2017). The participation gap. Social status and political inequality. Oxford University Press. doi: 10.1093/oso/9780198733607.001.0001

Dalton, R. (2008). Citizenship norms and the expansion of political participation. Political Studies, 56(1), 76--98. doi: \url{https://doi.org/10.1111/j.1467-9248.2007.00718.x}

Dalton, R. (2006). Citizen Politics. Public opinion and political parties in advanced industrial democracies. Washington D.C: CQ Press.

Dalton, R. J. \& Klingemann, H. D. (2007). Citizens and Political Behavior. En R. J. Dalton \& H. D. Klingemann (Eds.), The Oxford Handbook of Political Behavior (pp.~3--26). New York: Oxford University Press. doi: 10.1093/oxfordhb/9780199270125.001.0001

Dalton, R., van Sickle, A. \& Weldon, S. (2010). The individual--institutional nexus of protest behaviour. British Journal of Political Science, 40(1), 51--73. doi: \url{https://doi.org/10.1017/S000712340999038X}

Deimel, D., Hoskins, B. \& Abs, H. J. (2020). How do schools affect inequalities in political participation: compensation of social disadvantage or provision of differential access? Educational Psychology, 40(2), 146--166. doi: \url{https://doi.org/10.1080/01443410.2019.1645305}

Delamaza, G. (2010). La disputa por la participación en la democracia elitista chilena. Latin American Research Review, 45(4), 274--297. doi: \url{https://doi.org/10.1353/lar.2010.0038}

Della Porta, D. (2013). Can Democracy Be Saved: Participation, Deliberation and Social Movements. Cambridge: Polity.

Di Battista, S., Pivetti, M. \& Berti, C. (2014). Engagement in the university context: exploring the role of a sense of justice and social identification. Social Psychology of Education, 17(3), 471--490. doi: \url{https://doi.org/10.1007/s11218-014-9255-9}

Donoso, S. \& von Bülow, M. (Eds.) (2017). Social Movements in Chile. Organization, trajectories \& political consequences. Nueva York: Palgrave Macmillan. doi: 10.1057/978-1-137-60013-4

Dubet, F. (2011). Egalité des places, égalité des chances. Études, 414(1), 31--41. Recuperado de \url{https://www.cairn.info/revue-etudes-2011-1-page-31.htm\#}

Dubrow, J. K., Slomczynski, K. M. \& Tomescu-Dubrow, I. (2008). Effects of democracy and inequality on soft political protest in Europe: Exploring the European social survey data. International Journal of Sociology, 38(3), 36--51. doi: \url{https://doi.org/10.2753/IJS0020-7659380302}

Eckstein, K. \& Noack, P. (2016). Classroom climate effects on adolescents' orientations toward political behaviors: a multilevel approach. En P. Thijssen, J. Siongers, J. van Laer, J. Haers \& S. Mels (Eds.), Political engagement of the young in Europe: Youth in the crucible (pp.~161--177). London: Routledge.

Ehrhardt-Madapathi, N., Pretsch, J. \& Schmitt, M. (2018). Effects of Injustice in Primary Schools on Students' Behavior and Joy of Learning. Social Psychology of Education, 21(2), 337--369. doi: \url{https://doi.org/10.1007/s11218-017-9416-8}

Ekman, J. \& Amnå, E. (2012). Political participation and civic engagement: Towards a new typology. Human Affairs, 22(3), 283--300. doi: \url{https://doi.org/10.2478/s13374-012-0024-1}

Escobar, S. (2019). Participación juvenil: descripciones e interrogantes desde los resultados de la IX Encuesta Nacional de Juventud. Revista RT, 8, 7--13. Recuperado de \url{http://www.injuv.gob.cl/storage/docs/RT_31_WEB.pdf}

Fdez. Enguita, M. (1990). La escuela a examen (Un análisis sociológico para educadores y otras personas interesadas). Madrid: Universidad Complutense.

Finch, W. H., Bolin, J. E. \& Kelley, K. (2014). Multilevel Modeling Using R. New York: Chapman and Hall/CRC. Recuperado de \url{https://juancarloscastillo.github.io/jc-castillo/documents/diplomado_multinivel/bibliografia/finch\%20et\%20al\%202014\%20multilevel\%20modeling\%20using\%20R.pdf}

Flanagan, C. (2016). Epilogue. Youth and the `social contract'. En P. Thijssen, J. Siongers, J. van Laer, J. Haers \& S. Mels (Eds.), Political engagement of the young in Europe: Youth in the crucible (pp.~195--203). London: Routledge.

Flanagan, C. (2013) Teenage Citizens: The Political Theories of the Young. Cambridge: Harvard University Press.

Flanagan, C. A., Cumsille, P., Gill, S. \& Gallay, L. S. (2007). School and community climates and civic commitments: Processes for ethnic minority and majority students. Journal of Educational Psychology, 99(2), 421--431. doi: \url{https://doi.org/10.1037/0022-0663.99.2.421}

Flanagan, C. \& Sherrod, L. (1998). Youth political development: An introduction. Journal of Social Issues, 54(3), 447--456. doi: \url{https://doi.org/10.1111/j.1540-4560.1998.tb01229.x}

Flores, I., Sanhueza, C., Atria, J. \& Mayer, R. (2019). Top Incomes in Chile: A Historical Perspective on Income Inequality, 1964-2017. Review of Income and Wealth, (solo en línea). doi: \url{https://doi.org/10.1111/roiw.12441}

Freire, P. (2014). Pedagogía del Oprimido. Buenos Aires: Siglo XXI Editores Argentina.

Gainous, J. \& Martens, A. M. (2012). The effectiveness of civic education: Are ``good'' teachers actually good for ``all'' students? American Politics Research, 40(2), 232--266. doi: \url{https://doi.org/10.1177/1532673X11419492}

Garretón, M. A. (2016). La ruptura entre política y sociedad. Una introducción. En M. A. Garretón (Ed.), La gran ruptura. Institucionalidad política y actores sociales en el Chile del siglo XXI (pp.~11--20). Santiago de Chile: LOM Ediciones.

Garretón, M. A. \& Garretón, R. (2010). La democracia incompleta en Chile: La realidad tras los rankings internacionales. Revista de ciencia política, 30(1), 115--148. doi: \url{http://dx.doi.org/10.4067/S0718-090X2010000100007}

Garretón, M. A., Joignant, A., Somma, N. \& Campos, T. (Eds.) (2017). Conflicto Social en Chile 2015- 2016: disputando mitos. Santiago de Chile: Centro de Estudios de Conflictos y Cohesión Social. Recuperado de \url{https://coes.cl/publicaciones/n4-conflicto-social-en-chile-2015-2016-disputando-mitos/}

González, C. (2018). Cultura cívica escolar: propuesta de modelo conceptual para el estudio de formación ciudadana en espacios escolares (Tesis de magíster). Pontificia Universidad Católica de Chile, Santiago de Chile. Recuperado de \url{https://repositorio.uc.cl/handle/11534/22253}

Gorard, S. (2011). The potential determinants of young people's sense of justice: an international study. British Journal of Sociology of Education, 32(1), 35--52. doi: \url{https://doi.org/10.1080/01425692.2011.527721}

Gouveia‐Pereira, M., Vala, J. \& Correia, I. (2017). Teachers' legitimacy: Effects of justice perception and social comparison processes. British Journal of Educational Psychology, 87(1), 1--15. doi: \url{https://doi.org/10.1111/bjep.12131}

Hess, D. \& McAvoy, P. (2015). The political classroom: Evidence and ethics in democratic education. New York: Routledge.

Hoskins, B., Janmaat, J. G. \& Melis, G. (2017). Tackling inequalities in political socialisation: A systematic analysis of access to and mitigation effects of learning citizenship at school. Social Science Research, 68, 88--101. doi: \url{https://doi.org/10.1016/j.ssresearch.2017.09.001}

Hox, J. (2010). Multilevel analysis: Techniques and applications. New York: Routledge. Recuperado de \url{https://faculty.psau.edu.sa/filedownload/doc-12-pdf-7a03db1c2b00f1fac2f67c8e4e57414b-original.pdf}

Instituto Nacional de la Juventud (INJUV). (2019). Desarrollo Cívico/Social. En INJUV, 9na Encuesta Nacional de Juventud (pp.~56--85). Santiago de Chile: Ministerio de Desarrollo Social y Familia. Recuperado de \url{http://www.injuv.gob.cl/noticias/9encuesta}

Instituto Nacional de la Juventud (INJUV). (2017). Democracia y Participación Sociopolítica de las Juventudes. En INJUV, 8va Encuesta Nacional de Juventud (pp.~66--93). Santiago de Chile: Ministerio de Desarrollo Social. Recuperado de \url{http://www.injuv.gob.cl/storage/docs/Libro_Octava_Encuesta_Nacional_de_Juventud.pdf}

Isac, M, Maslowski, R. Creemers, B. \& van der Werf, G. (2014). The contribution of schooling to secondary-school students' citizenship outcomes across countries. School Effectiveness and School Improvement, 25(1), 29--63. doi: \url{https://doi.org/10.1080/09243453.2012.751035}

Jasso, G. \& Resh, N. (2002). Exploring the Sense of Justice about Grades. European Sociological Review, 18(3), 333--351. doi: \url{https://doi.org/10.1093/esr/18.3.333}

Kaplan, C. V. \& Ferrero, F. (2003). Los ganadores y los perdedores. Un examen de la noción de talentos naturales asociada con el éxito o fracaso escolar. Educación, lenguaje y sociedad, 1(1), 121--136. Recuperado de \url{http://www.biblioteca.unlpam.edu.ar/pubpdf/ieles/n01a08kaplan.pdf}

Keating, A. \& Janmaat, J. G. (2016). Education through citizenship at school: Do school activities have a lasting impact on youth political engagement? Parliamentary Affairs, 69(2), 409--429. doi: \url{https://doi.org/10.1093/pa/gsv017}

Kudrnáč, A. (2017). Vliv klimatu školní třídy a jejího socioekonomického složení na občanské znalosti a postoj k volební účasti. Sociologický časopis, 53(2), 209--240. doi: \url{https://doi.org/10.13060/00380288.2017.53.2.314}

Lampert, K. (2013). Meritocratic education and social worthlessness. New York: Palgrave Pivot. doi: \url{https://doi.org/10.1057/9781137324894}

Lijphart, A. (1997). Unequal participation: democracy's unresolved dilemma. American Political Science Review, 91(1), 1--14. doi: 10.2307 / 2952255

Lechner, N. (1994). Chile 2000: Las sombras del mañana. Estudios Internacionales, 27(105), 3-11. doi: 10.5354/0719-3769.2011.15352

Lenzi, M., Vieno, A., Sharkey, J., Mayworm, A., Scacchi, L., Pastore, M. \& Santinello, M. (2014). How School can Teach Civic Engagement Besides Civic Education: The Role of Democratic School Climate. American Journal of Community Psychology, 54, 251--261. doi: \url{https://doi.org/10.1007/s10464-014-9669-8}

Luna, J. P. (2016). Delegative democracy revisited. Chile's crisis of representation. Journal of Democracy, 27(3), 129--138. doi: \url{https://doi.org/10.1353/jod.2016.0046}

Mameli, C., Biolcati, R., Passini, S. \& Mancini, G. (2018). School context and subjective distress: The influence of teacher justice and school-specific well-being on adolescents' psychological health. School Psychology International, 39(5), 526--542. doi: \url{https://doi.org/10.1177/0143034318794226}

Manganelli, S., Lucidi, F. \& Alivernini, F. (2015). Italian adolescents' civic engagement and open classroom climate: The mediating role of self-efficacy. Journal of Applied Developmental Psychology, 41, 8--18. doi: \url{https://doi.org/10.1016/j.appdev.2015.07.001}

Marcin, K., Morinaj, J. \& Hascher, T. (2019). The Relationship between Alienation from Learning and Student Needs in Swiss Primary and Secondary Schools. Zeitschrift für Pädagogische Psychologie, 34(1), 35--49. doi: \url{https://doi.org/10.1024/1010-0652/a000249}

Marien, S., Hooghe, M. \& Quintelier, E. (2010). Inequalities in non-institutionalised forms Of political participation: A multi-level analysis of 25 countries. Political Studies, 58(1), 187--213. doi: \url{https://doi.org/10.1111/j.1467-9248.2009.00801.x}

Medel, R. \& Somma, N. (2016). ¿Marchas, ocupaciones o barricadas? Explorando los determinantes de las tácticas de la protesta en Chile. Política y gobierno, 23(1), 163--199. Recuperado de \url{http://www.scielo.org.mx/scielo.php?script=sci_arttext\&pid=S1665-20372016000100163}

Ministerio de Educación (MINEDUC), División de Educación General. (2016). Orientaciones para la elaboración del Plan de Formación Ciudadana. Recuperado de \url{http://formacionciudadana.mineduc.cl/wp-content/uploads/sites/46/2016/04/DEG-OrientacionesPFC-intervenible-AReader_FINAL.pdf}

Miranda, D. (2018). Desigualdad y ciudadanía: una aproximación intergeneracional (Tesis de doctorado). Pontificia Universidad Católica de Chile, Santiago de Chile. Recuperado de \url{https://repositorio.uc.cl/handle/11534/22255}

Miranda, D., Castillo, J. C. \& Sandoval-Hernandez, A. (2017). Young Citizens Participation: Empirical Testing of a Conceptual Model. Youth \& Society, 52(2), 251--271. doi: \url{https://doi.org/10.1177/0044118X17741024}

Mizala, A., \& Torche, F. (2012). Bringing the schools back in: the stratification of educational achievement in the Chilean voucher system. International Journal of Educational Development, 32(1), 132--144. doi: \url{https://doi.org/10.1016/j.ijedudev.2010.09.004}

Navia, P. (2004). Participación electoral en Chile, 1988-2001. Revista de ciencia política, 24(1), 81-103. doi: \url{http://dx.doi.org/10.4067/S0718-090X2004000100004}

Neundorf, A., Niemi, R. G. \& Smets, K. (2016). The compensation effect of civic education on political engagement: How civics classes make up for missing parental socialization. Political Behavior, 38(4), 921--949. doi: \url{https://doi.org/10.1007/s11109-016-9341-0}

Neundorf, A. \& Smets, K. (2017). Political socialization and the making of citizens. Oxford Handbooks Online. \url{doi:10.1093/oxfordhb/9780199935307.013.98}

Núñez, C., Osorio, R. \& Peit, M. (2018). ¿Empoderamiento ciudadano? Evolución y determinantes de la participación y la politización en Chile, 1990-2016. En C. Berríos \& C. García (Eds.), Ciudadanías en conflictos. Enfoques, experiencias y propuestas (pp.~223--259). Santiago de Chile: Ariadna Ediciones. Recuperado de \url{https://books.openedition.org/ariadnaediciones/1214?lang=es}

Ortiz-Inostroza, C. \& Lopez, E. (2017). Explorando modelos estadísticos para explicar la participación en protestas en Chile. Revista de Sociología 32(1), 13--31. doi: 10.5354/0719-529x.2017.47883

Pansu, P., Dubois, N. \& Dompnier, B. (2008). Internality-norm theory in educational contexts. European Journal of Psychology of Education, 23(4), 385--397. doi: \url{https://doi.org/10.1007/BF03172748}

Parker, C. (2000). Los jóvenes chilenos. Cambios culturales, perspectivas del siglo XXI. Santiago de Chile: Mideplan. Recuperado de \url{https://www.academia.edu/13187022/Los_j\%C3\%B3venes_chilenos_cambios_culturales_perspectivas_para_el_siglo_XXI}

Peña, C. (2015). Escuela y vida cívica. En C. Cox \& J. C. Castillo (Eds.), Aprendizaje de la ciudadanía: Contextos, experiencias y resultados (pp.~27--49). Santiago de Chile: Ediciones Universidad Católica de Chile.

Peña, M. \& Toledo, C. (2016). Discursos sobre clase social y meritocracia de escolares vulnerables en Chile. Cadernos de Pesquisa, 47(164), 496--518. doi: \url{https://doi.org/10.1590/198053143752}

Peter, F. \& Dalbert, C. (2010). Do my teachers treat me justly? Implications of students' justice experience for class climate experience. Contemporary Educational Psychology, 35(4), 297--305. doi: \url{https://doi.org/10.1016/j.cedpsych.2010.06.001}

Pretsch, J. \& Ehrhardt-Madapathi, N. (2018). Experiences of justice in school and attitudes towards democracy: A matter of social exchange? Social Psychology of Education, 21(1), 655--675. doi: \url{https://doi.org/10.1007/s11218-018-9435-0}

Programa de las Naciones Unidas para el Desarrollo (PNUD). (2019). Diez años de auditoría a la democracia: Antes del estallido. Santiago de Chile: Programa de las Naciones Unidas para el Desarrollo. Recuperado de \url{https://www.cl.undp.org/content/chile/es/home/library/diez-anos-de-auditoria-a-la-democracia--antes-del-estallido.html}

Programa de las Naciones Unidas para el Desarrollo (PNUD). (2017a). Diagnóstico sobre la participación electoral en Chile. Santiago de Chile: Programa de las Naciones Unidas para el Desarrollo. Recuperado de \url{https://www.cl.undp.org/content/chile/es/home/library/democratic_governance/diagnostico-sobre-la-participacion-electoral-en-chile.html}

Programa de las Naciones Unidas para el Desarrollo (PNUD). (2017b). Desiguales: Orígenes, cambios y desafíos de la brecha social en Chile. Santiago de Chile: Programa de Naciones Unidas para el Desarrollo. Recuperado de \url{https://www.cl.undp.org/content/chile/es/home/library/poverty/desiguales--origenes--cambios-y-desafios-de-la-brecha-social-en-.html}

Programa de las Naciones Unidas para el Desarrollo (PNUD). (2015). Informe sobre Desarrollo Humano en Chile 2015. Los tiempos de la politización. Santiago de Chile: Programa de Naciones Unidas para el Desarrollo. Recuperado de \url{https://www.cl.undp.org/content/chile/es/home/library/human_development/los-tiempos-de-la-politizacion.html}

Quintelier, E. (2010). The effect of schools on political participation: A multilevel logistic analysis. Research Papers in Education, 25(2), 137--154. doi: \url{https://doi.org/10.1080/02671520802524810}

Quintelier, E. \& Blais, A. (2016). Intended and Reported Political Participation. International Journal of Public Opinion Research, 28(1), 117--128. doi: \url{https://doi.org/10.1093/ijpor/edv017}

Quintelier, E. \& Hooghe, M. (2013). The relationship between political participation intentions of adolescents and a participatory democratic climate at school in 35 countries. Oxford Review of Education, 39(5), 567--589. doi: \url{https://doi.org/10.1080/03054985.2013.830097}

Resh, N. (2018). Sense of Justice in School and Social and Institutional Trust. Comparative Sociology, 14(3-4), 369--385. doi: \url{https://doi.org/10.1163/15691330-12341465}

Resh, N. (2010). Sense of justice about grades in school: is it stratified like academic achievement? Social Psychology of Education, 13(3), 313--329. doi: \url{https://doi.org/10.1007/s11218-010-9117-z}

Resh, N. \& Dalbert, C. (2007). Gender Differences in Sense of Justice about Grades: A Comparative Study of High School Students in Israel and Germany. Teachers College Record, 109(2), 322--342. Recuperado de \url{https://www.researchgate.net/profile/Nura_Resh/publication/286356463_Gender_differences_in_sense_of_justice_about_grades_A_comparative_study_of_high_school_students_in_Israel_and_Germany/links/5bad27e845851574f7ea9643/Gender-differences-in-sense-of-justice-about-grades-A-comparative-study-of-high-school-students-in-Israel-and-Germany.pdf}

Resh, N. \& Sabbagh, C. (2017). Sense of justice in school and civic behavior. Social Psychology of Education, 20(2), 387--409. doi: \url{https://doi.org/10.1007/s11218-017-9375-0}

Resh, N. \& Sabbagh, C. (2016). Justice and Education. En C. Sabbagh \& M. Schmitt (Eds.), Handbook of Social Justice Theory and Research (pp.~349--368). Springer US. doi: 10.1007/978-1-4939-3216-0

Resh, N. \& Sabbagh, C. (2014). Sense of justice in school and civic attitudes. Social Psychology of Education, 17 (1), 51--72. doi: \url{https://doi.org/10.1007/s11218-013-9240-8}

Resh, N. \& Sabbagh, C. (2013). Justice, belonging and trust among Israeli middle school students. British Educational Research Journal, 40(6), 1036--1056. doi: \url{https://doi.org/10.1002/berj.3129}

Resh, N. \& Sabbagh, C. (2009). Justice in Teaching. En L. Saha \& A. G. Dworkin (Eds.), International Handbook of Research on Teachers and Teaching (pp.~669--682). Springer US. doi: 10.1007/978-0-387-73317-3

Roberts, K. M. (2016). (Re)Politicizing Inequalities: Movements, Parties, and Social Citizenship in Chile. Journal of Politics in Latin America, 8(3), 125--154. doi: \url{https://doi.org/10.1177/1866802X1600800305}

Rozas, J. \& Somma, N. (2020). Determinantes de la protesta juvenil en Chile. Revista Mexicana de Sociología, 82(3), 673--703. doi: \url{http://dx.doi.org/10.22201/iis.01882503p.2020.3.58506}

Ruiz, C. (2015). De nuevo la sociedad. Santiago de Chile: LOM Ediciones.

Ruiz, C. \& Boccardo, G. (2014). Los chilenos bajo el neoliberalismo. Clases y conflicto social. Santiago de Chile: Nodo XXI--Desconcierto.

Ruiz, C. \& Caviedes, S. (2020a). Estructura y conflicto social en la crisis del neoliberalismo avanzado chileno. Espacio Abierto, 29(1), 86--101. Recuperado de \url{http://repositorio.uchile.cl/handle/2250/175352}

Ruiz, C. \& Caviedes, S. (2020b). La rebelión de los hijos de la modernización neoliberal. En R. Baño, H. Fazio, A. Mayol \& C. Ruiz (2020), Análisis del año 2019 (pp.~27--47). Santiago de Chile: Departamento de Sociología Universidad de Chile.

Sabbagh, C. \& Resh, N. (2016). Unfolding Justice Research in the Realm of Education. Social Justice Research, 29(1), 1--13. doi: \url{https://doi.org/10.1007/s11211-016-0262-1}

Sabbagh, C., Resh, N., Mor, M. \& Vanhuysse, P. (2006). Spheres of Justice within Schools: Reflections and Evidence on the Distribution of Educational Goods. Social Psychology of Education, 9(2), 97--118. doi: \url{https://doi.org/10.1007/s11218-005-3319-9}

Sampermans, D. \& Claes, E. (2018). Teachers as role models in the political socialization process: How a good student--teacher relationship can compensate for gender differences in students' gender equality attitudes. Citizenship Teaching and Learning, 13(1), 105--125. doi: \url{https://doi.org/10.1386/ctl.13.1.105_1}

Sandoval, J. \& Carvallo, V. (2019). Una generación «sin miedo»: análisis de discurso de jóvenes protagonistas del movimiento estudiantil chileno. Ultima década, 27(51), 225--257. doi: \url{https://dx.doi.org/10.4067/S0718-22362019000100225}

Sandoval, J. \& Carvallo, V. (2017). Discursos sobre política y democracia de estudiantes universitarios chilenos de distintas organizaciones juveniles. Revista Española de Ciencias Políticas, 43, 137--160. doi: \url{https://doi.org/10.21308/recp.43.06}

Santos, H. \& Elacqua, G. (2016). Segregación socioeconómica escolar en Chile: elección de la escuela por los padres y un análisis contrafactual teórico. Revista CEPAL, 119, 133--148. Recuperado de \url{https://repositorio.cepal.org/handle/11362/40396}

Schlozman, K. L., Verba, S. \& Brady, H. E. (2012). The Unheavenly Chorus: Unequal Political Voice and the Broken Promise of American Democracy. Princeton: Princeton University Press. doi: 10.2307/j.ctt7sn9z

Siavelis, P. M. (2016). Crisis of representation in Chile? Journal of Politics in Latin America, 8(3), 61--93. Recuperado de \url{https://journals.sub.uni-hamburg.de/giga/jpla/article/view/1004/1011}

Somma, N. (2017). Protestas y conflictos en el Chile contemporáneo: Quince tesis para la discusión. En R. Araya \& F. Ceballos (Eds.), Conflictos, controversias y disyuntivas (pp.~37--85). Santiago de Chile: TIRONI Asociados. Recuperado de \url{http://sociologia.uc.cl/wp-content/uploads/2017/08/somma-2017-quince-tesis-conflicto-y-protesta.pdf}

Somma, N. \& Bargsted, M. (2018). Political inequality in 38 countries: A distributional approach. Comparative Sociology, 17(5), 469--495. doi: \url{https://doi.org/10.1163/15691330-12341475}

Somma, N. \& Bargsted, M. (2015). La Autonomización de la Protesta en Chile. En C. Cox \& J. C. Castillo (Eds.), Aprendizaje de la ciudadanía: Contextos, experiencias y resultados (pp.~209--240). Santiago de Chile: Ediciones Universidad Católica de Chile.

Somma, N. Bargsted, M., \& Sánchez, F. (2020). Protest Issues and Political Inequality in Latin America. American Behavioral Scientist. doi: \url{https://doi.org/10.1177/0002764220941233}

Somma, N. \& Medel, R. (2017). Shifting Relationships Between Social Movements and Institutional Politics. En S. Donoso \& M. von Bülow (Eds.), Social movements in Chile. Organization, trajectories \& political consequences (pp.~29--61). New York: Palgrave Macmillan.

Tarabini, A. (2015). La meritocracia en la mente del profesorado: un análisis de los discursos docentes en relación al éxito, fracaso y abandono escolar. Revista de la Asociación de Sociología de la Educación, 8(3), 349--360. Recuperado de \url{https://ojs.uv.es/index.php/RASE/article/view/8389/7982}

Tateo, L. (2018). Ideology of Success and the Dilemma of Education Today. En A. Joerchel \& G. Benetka (Eds.), Memories of Gustav Ichheiser. Theory and History in the Human and Social Sciences (pp.~157--164). Springer, Cham. doi: \url{https://doi.org/10.1007/978-3-319-72508-6_9}

Toro, S. (2008). De lo épico a lo cotidiano: Jóvenes y generaciones políticas en Chile. Revista de ciencia política, 28(3), 143--160. doi: \url{http://dx.doi.org/10.4067/S0718-090X2008000200006}

Toro, S. (2007). La inscripción electoral de los jóvenes en Chile. Factores de incidencia y aproximaciones al debate. En A. Fontaine, C. Larroulet, J. A. Viera-Gallo \& I. Walker, Modernización del régimen electoral chileno (pp.~101--122). Santiago de Chile: Programa de las Naciones Unidas para el Desarrollo. Recuperado de \url{https://www.cieplan.org/wp-content/uploads/2019/12/Modernizacion-del-regimen-electoral-chileno.pdf}

Treviño, E., Villalobos, C., Béjares, C. \& Naranjo, E. (2019). Forms of Youth Political Participation and Educational System: The Role of the School for 8th Grade Students in Chile. YOUNG, 27(3), 279--303. doi: \url{https://doi.org/10.1177/1103308818787691}

Valenzuela, J. P., Bellei, C. \& de los Ríos, D. (2014). Socioeconomic school segregation in a market-oriented educational system. The case of Chile. Journal of Education Policy, 29(2), 217--241. doi: \url{https://doi.org/10.1080/02680939.2013.806995}

Van Deth, J. W. (2014). A conceptual map of political participation. Acta Polit 49(3), 349--367. doi: \url{https://doi.org/10.1057/ap.2014.6}

Verba, S., Schlozman, K. L. \& Brady, H. E. (1995). Voice and Equality. Civic Voluntarism in American Politics. Londres: Harvard University Press.

Walzer, M. (1983). Spheres of justice. New York: Basic Books.

Wanders, F. H. K., Dijkstra, A. B., Maslowski, R. \& van der Veen, I. (2020). The effect of teacher-student and student-student relationships on the societal involvement of students. Research Papers in Education, 35(2), 1--21. doi: \url{https://doi.org/10.1080/02671522.2019.1568529}

Wanders, F. H. K., van der Veen, I., Dijkstra, A. B. \& Maslowski, R. (2020). The influence of teacher-student and student-student relationships on societal involvement in Dutch primary and secondary schools. Theory \& Research in Social Education, 48(1), 101--119. doi: \url{https://doi.org/10.1080/00933104.2019.1651682}

Youniss, J. \& Levine, P. (2009). Engaging young people in civic life. Nashville: Vanderbilt University Press.

Zarzuri, R. (2016). Las transformaciones en la participación política de los jóvenes en el Chile actual. En M. A. Garretón (Ed.), La gran ruptura. Institucionalidad política y actores sociales en el Chile del siglo XXI (pp.~133--160). Santiago de Chile: LOM Ediciones.

% %%%%%%%%%%%%%%%%%%%%%%%%%%%%%%%%%%%%%%%%%%%%%%%%%
% %%% Bibliography                              %%%
% %%%%%%%%%%%%%%%%%%%%%%%%%%%%%%%%%%%%%%%%%%%%%%%%%
% \addtocontents{toc}{\vspace{.5\baselineskip}}
% \cleardoublepage
% \phantomsection
% \addcontentsline{toc}{chapter}{\protect\numberline{}{Bibliography}}
\bibliography{tesis}

%% All books from our library (SfS) are already in a BiBTeX file
%% (Assbib). You can use Assbib combined with your personal BiBTeX file:
%% \bibliography{Myreferences,Assbib}. Of course, this will only work on
%% the computers at SfS, unless you copy the Assbib file
%%  --> /u/sfs/bib/Assbib.bib



\end{document}
